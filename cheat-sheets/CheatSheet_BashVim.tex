% Dokumenten-Einstellungen
\documentclass[a4paper, twocolumn]{scrarticle}
\usepackage[left=1.5cm,right=1.5cm,top=1.5cm,bottom=1.5cm]{geometry}

% Spracheinstellungen
\usepackage[utf8]{inputenc}
\usepackage[T1]{fontenc}
\usepackage[ngerman]{babel}

% Standard-Schrift auf footnoteisze  und serifenlos setzen
\renewcommand{\normalsize}{\footnotesize}
\AtBeginDocument{\normalsize}
\renewcommand{\familydefault}{\sfdefault}

% Caption-Nummerierung soll fettgedruckt sein
\usepackage[labelfont=bf]{caption}

\setlength{\parindent}{0pt} % Kein Einruecken beim Absatz
\setlength{\columnsep}{24pt} % Abstand zwischen den Spalten

% Syntax-Highlighting und Code-Umgebung
\usepackage{xcolor}  % für Farben
\usepackage{listings}  % für Code-Blöcke
\lstdefinelanguage{Bash}{
  keywords={if, then, else, fi, for, while, do, done, echo, exit, function, case, esac},
  keywords=[2]{sed, grep, cut, head, tail, ssh, scp},
  sensitive=true,    % Bash ist case-sensitive
  moredelim=[s][\color{purple}]{<}{> \ },  % Platzhalter in spitzen Klammern faerben
  morecomment=[l]{\#},   % Kommentare starten mit #
  morestring=[b]{"},      % Strings in " "
  morestring=[b]{'},      % Strings in ' '
  keywordstyle=\color{orange}\bfseries,
  keywordstyle=[2]\color{blue}\bfseries,
  commentstyle=\color{gray},
  stringstyle=\color{olive},
  basicstyle=\ttfamily\small,
  showstringspaces=true
}
% Stil für Makefile-Syntax
\lstdefinelanguage{Makefile}{
  morekeywords={all, clean, install, uninstall, .PHONY, include, export, ifeq, else, endif, define, endef},
  morekeywords=[2]{CC, CFLAGS, LDFLAGS, RM, AR, AS, LD, MAKE, MAKEFLAGS}, % Typische Variablen
  moredelim=[s][\color{purple}]{<}{> \ },  % Platzhalter in spitzen Klammern faerben
  morecomment=[l]{\#},    % Kommentare mit #
  morestring=[b]{"},        % Strings in " "
  morestring=[b]{'},        % Strings in ' '
  keywordstyle=\color{blue},
  keywordstyle=[2]\color{orange}, % Variablen in einer anderen Farbe
  commentstyle=\color{gray},
  stringstyle=\color{purple},
  basicstyle=\ttfamily\small,
  showstringspaces=false,
  tabsize=4
}
% Stil für Vim-Befehle
\lstdefinelanguage{Vim}{
  morekeywords={q, @, v, V, I, A, s, g, !},
  morecomment=[l]{\"},     % Vim-Kommentare starten mit "
  morestring=[b]',         % Strings in ' '
  morestring=[b]",         % Strings in " "
  moredelim=[s][\color{purple}]{<}{>},  % Platzhalter in spitzen Klammern faerben
  moredelim=[s][\color{orange}]{[}{]},
  keywordstyle=\color{blue}\bfseries,
  commentstyle=\color{gray},
  stringstyle=\color{purple},
  basicstyle=\ttfamily\small,
  showstringspaces=false
}

\lstset{
  language=Bash,
  basicstyle=\ttfamily\footnotesize,
  frame=single,
  xleftmargin=-6pt,
  framexleftmargin=-8pt,
  numbers=left,
  numberstyle=\tiny\color{gray},
  numbersep=-2pt,
  stepnumber=1,
  breaklines=true,
  tabsize=4,
  showstringspaces=false,
  captionpos=t
}

% Listing soll in Caption als 'Code' bezeichnet werden
\renewcommand{\lstlistingname}{Code}

% INLINE: \lstinline|<code>|
% UMGEBUNG: \begin{lstlisting}[caption={[<shortcaption>]<Caption>}, label=<label>] <codeblock> \end{lstlisting}

\usepackage{hyperref}
\hypersetup{
  colorlinks=true,
  linkcolor=purple,
  urlcolor=blue,
  pdftitle={CheatSheet Bash, Make \& Vim}
}
\usepackage{cleveref}
\crefname{section}{Abschnitt}{Abschnitte}
\crefname{listing}{Code}{Codes}

\begin{document}

% Überschrift
{\Huge  \textbf{\textsf{Bash, Make \& Vim Cheat Sheet}}} \ {\tiny (sehr unvollständig!)}\\
Zusammenstellung: Karl Hänsel, Stand: \today

\section{Bash}
\subsection{Dateien und Ordner}
\textbf{Symbolischer Link:} \lstinline|ln -s <quelle> <ziel>|\\
\textbf{Metadaten aktualisieren oder neue, leere Datei erstellen:}\\
\lstinline|touch <Datei>|

\subsection{Pipes und Dateiausgabe}
\textbf{Ausgabe eines Befehls als Argument eines anderen nutzen:} \\ \lstinline!<befehl1> <optionen1> <argumente1> | <befehl2>...!\\
\textbf{Ausgabe eines Befehls in Datei umleiten:}\\
\lstinline|<befehl> <optionen> <argumente> <Datei> |\\
\textit{Pipes können mit Pipes sowie (mehrere) Pipes mit Dateiumleitung kombiniert werden!}

\subsection{Dateien bearbeiten}
\subsubsection{sed: Suchen, Ersetzen, Löschen, Einfügen}
\textbf{CAVE:} Option \lstinline|-i| nimmt sofort alle Änderungen an Datei vor, vorher lieber einmal ohne überprüfen (dann nur Ausgabe der Änderungen in Konsole)\\
\textbf{In Datei global suchen \& ersetzen:}\\\lstinline|sed -i 's/<such>/<ersetz>/'| \lstinline|<Datei>|\\
\textbf{In bestimmten Zeile(n) suchen \& ersetzen:}\\ \lstinline|sed -i '<n>s/<such>/<erset/'| \lstinline|<Datei>|, wobei \lstinline|<n>| eine Zeilennummer oder ein Bereich mit \lstinline|<start>,<end>| sein kann\\
\textbf{In Zeilen mit einem Muster suchen \& ersetzen:}\\ \lstinline|sed -i '/<Muster>/s/<such>/<ersetz>/'| \lstinline|<Datei>|\\
\textbf{Zeilen löschen:} alles wie oben nur \lstinline|d| statt \lstinline|s|\\
\textbf{Text einfügen:} nach Zeile \lstinline|n|: \lstinline|sed -i '<n>a\<text>'| \lstinline|<Datei>|, vor Zeile \lstinline|n|: \lstinline|sed -i '<n>i\<text>'| \lstinline|<Datei>|\\
\textbf{Mehrere Änderungen:} mehrere Befehlszeichenketten nach \lstinline|sed|
\subsubsection{grep: Muster in Zeilen finden}
\textbf{Alle Zeilen mit Muster:} \lstinline|grep '<Muster>'| \lstinline|<Datei>|\\
\textbf{Alle Zeilen OHNE Muster:} \lstinline|grep -v '<Muster>'| \lstinline|<Datei>|\\
\textbf{Alle Dateien mit Muster:} \lstinline|grep -r '<Muster>'| \lstinline|<Ordner>|\\
\textbf{Weitere Optionen:} \lstinline|-o| nur MIT Muster ausgeben (Standard),\\ \lstinline|-i| Groß-/Kleinschreibung ignorieren, \lstinline|-n| Zeilennummer anzeigen, \lstinline|-E| erweiterte reguläre Ausdrücke für Muster verwenden
\subsubsection{cut: Spalten extrahieren}
\textbf{Bestimmte Felder (Spalten) extrahieren mit Tab-Trennung:} \lstinline|cut -f <n> <Datei>|, dabei ist \lstinline|<n>| Spaltennummer, mehrere Spaltennummern \lstinline|<n1,n2>| oder Bereich \lstinline|<n1-n2>|\\
\textbf{Anderes Trennzeichen wählen:}\\ \lstinline|cut -d '<Trennzeichen>'| \lstinline|-f <n> <Datei>|\\
\textbf{Bestimmte Zeichen pro Zeile extrahieren:}\\
\lstinline|cut -c <n>| (dabei \lstinline|<n>| wie oben)
\subsubsection{head \& tail: Anfang \& Ende extrahieren}
\textbf{Erste 10 Zeilen ausgeben:} \lstinline|head <Datei>|\\
\textbf{Erste \lstinline|n| Zeilen ausgeben:} \lstinline|head -n <n> <Datei>|\\
\textbf{Erste \lstinline|n| Zeichen ausgeben:} \lstinline|head -c <n> <Datei>|\\
\textbf{Letzte 10 Zeilen ausgeben:} \lstinline|tail <Datei>|\\
\textbf{Letzte \lstinline|n| Zeilen ausgeben:} \lstinline|tail -n <n> <Datei>|\\
\textbf{Echtzeitanzeige der neuesten Einträge einer Datei:} \\ \lstinline|tail -f <Datei>| (z.B. für Log-Dateien)


\subsection{Platzhalter}
\textbf{Aktueller Ordner:} \lstinline|.|\\
\textbf{Ebene über aktuellem Ordner:} \lstinline|..|\\
\textbf{Alle Dateien in einem Ordner:} \lstinline|*|\\
\textbf{Alle Dateien mit bestimmter Endung:} \lstinline|*.<endung>|

\subsection{Befehl wiederholen (while true)}\label{sec:bash-whiletrue}
\textbf{Befehl dauerhaft wiederholen} im Abstand von \lstinline|<integer>| Sekunden (bis Abbruch, z.B. durch Strg+C):\\ \lstinline|while true; do <befehl>; sleep <integer>; done|\\
(am einfachsten in separatem Tab/Konsolenfenster)

\subsection{SSH (Secure Shell)}
\textbf{Remote-Zugriff auf Server in Konsole:}\\
\lstinline|ssh <Benutzer>@<Server>|\\
(anschließend Passworteingabe, Beenden mit \lstinline|exit|)\\
\textbf{Kopieren zwischen Remote und lokal:}\\
\lstinline|scp <Benutzer>@<Server>:<Remotepfad> <Lokalpfad>|\\
(oder umgekehrt für Lokal zu Remote)

\section{Make}
\lstset{language=Makefile}
\begin{lstlisting}[caption={\bfseries Vorlage \LaTeX-Makefile (Tab-Einrückungen wichtig!)}, label=code:make-latex]
  # Name Haupt-Datei (ohne .tex)
  DOC = main
  
  # Standardbefehl LaTex-Kompilierung
  LATEX = pdflatex
  BIBTEX = bibtex
  
  # Default-Target: PDF generieren
  all: $(DOC).pdf
  
  # PDF aus der Haupt-.tex
  $(DOC).pdf: $(DOC).tex
    $(LATEX) $(DOC).tex
    # nur ausgefuehrt falls Literatur:
    $(BIBTEX) $(DOC) || true
    $(LATEX) $(DOC).tex
    # noch mal fuer korrekte Referenzen
    $(LATEX) $(DOC).tex
  
  # clean entfernt Temporaerdateien
  clean:
    rm -f *.aux *.log *.out *.bbl *.blg *.toc *.lof *.lot *.nav *.snm *.vrb
  
  # fullcelan wie clean, aber auch PDF
  fullclean: clean
    rm -f $(DOC).pdf
    
\end{lstlisting}
\textbf{Verwendung Makefile (Beispiel \LaTeX):}\\
\LaTeX-Makefile als Datei \glqq Makefile\grqq im Projektverzeichnis speichern. Aufruf in Konsole über \lstinline|make| (wenn nicht mehrdeutig). Aufräumen temporärer Dateien dann mit \lstinline|make clean|. Temporäre Dateien und PDF löschen mit  \lstinline|make fullclean|. Alternativ möglich: eigenständiges Aufrufen des Befehls (aus vim heraus z.B. \lstinline|:!make|).

\section{Vi(m)}
\lstset{language=Vim}
{\scriptsize \textit{(wenn nicht anders angegeben Befehle im Normal Mode)}}\\
\textbf{Makro aufnehmen:} \lstinline|q<zeichen><befehle><zeichen>|\\
\textbf{Makro (\lstinline|n| mal) ausführen:} \lstinline|<n>@<zeichen>|\\
\textbf{Wechsel in Visual Mode (Markieren und anschließend auf Markierung Befehle anwenden):} zeichenweise: \lstinline|v|, zeilenweise: \lstinline|V|, blockweise: \lstinline|[STRG+]v| (in Windows bereits mit Einfügen belegt, dort: \lstinline|[STRG+ALT+]v|)\\
\textbf{In mehreren Zeilen Einfügen:} Auswahl der Zeilen mit Visual Line Mode (\lstinline|V|), dann \lstinline|I<text>[ENTER]| für Zeilenanfang \lstinline|A<text>[ENTER]| für Zeilenende\\
\textbf{Suchen und Ersetzen ganze Datei:} \lstinline|:%s/<such>/<ersetz>/g|\\
\textbf{Suchen und Ersetzen Markierung:} nach Markieren mit Visual Mode wird bei Eingabe von \lstinline|:| automatisch eine Zeichenkette ergänzt, die ein Ausüben des anschließenden Befehls Suchen und Ersetzen (wie oben) auf markierte Region bewirkt\\
\textbf{Konsolenbefehle nutzen:} \lstinline[language=Vim]|:!<befehl...>|
\end{document}
