\documentclass[a4paper,10pt, twocolumn]{scrarticle}
\usepackage[left=1.5cm,right=1.5cm,top=2cm,bottom=2cm]{geometry}
\usepackage[utf8]{inputenc}
\usepackage[T1]{fontenc}
\usepackage[ngerman]{babel}
\usepackage{hyperref}
\hypersetup{
  colorlinks=true,
  linkcolor=orange,
  citecolor=green,
  urlcolor=green
}

\setlength{\parindent}{0pt} % Kein Einrücken der Absätze
\setlength{\columnsep}{24pt} % Abstand zwischen den Spalten

% Syntax-Highlighting und Code-Umgebung
\usepackage{xcolor}
\usepackage{listings}
\lstdefinelanguage{Fortran95}{
  language=[95]Fortran,
  basicstyle=\ttfamily\small,
  keywordstyle=\color{blue},
  commentstyle=\color{gray},
  stringstyle=\color{purple},
  morekeywords={xor, kind, module, use, implicit, none, integer, real, double, precision, complex, logical, character, parameter, dimension, allocate, deallocate, contains, subroutine, function, end, if, then, else, endif, do, while, cycle, exit, select, case, default, write, read, print, format, open, close, rewind, backspace, inquire, stop, error, go, to, continue, return, call, data, save, common, equivalence, external, intrinsic, assign, pause, class, type, public, private, sequence, extends, import}
}

\lstset{
  language=Fortran95,
  frame=single,
  numbers=left,
  numberstyle=\tiny\color{gray},
  stepnumber=1,
  breaklines=true,
  tabsize=4,
  showstringspaces=false,
  captionpos=t,
  escapeinside={(*@}{@*)}  % Ermöglicht LaTeX-Befehle innerhalb von Listings
}

\lstdefinestyle{neutral}{
  language=bash,
  basicstyle=\ttfamily\small,
  numbers=left,
  numberstyle=\tiny\color{gray},
  backgroundcolor=\color{white},
  frame=single,
  rulecolor=\color{black},
  captionpos=b,
  commentstyle=\color{black},
  stringstyle=\color{black},
  showstringspaces=false,
  keywordstyle=\color{black}, 
  morekeywords={}
}

\renewcommand{\lstlistingname}{Code-Beispiel}

% INLINE: \lstinline|<code>|
% UMGEBUNG: \begin{lstlisting}[caption={[<shortcaption>]<Caption>}, label=<label>] <codeblock> \end{lstlisting}
% LaTeX-Befehl innerhalb Umgebung: (*@<LaTeX-Syntax>@*)

% Cleveref
\usepackage{cleveref}
\crefname{listing}{Code-Beispiel}{Code-Beispiele}
\Crefname{listing}{Code-Beispiel}{Code-Beispiele}


\begin{document}
  
  {\Huge \textbf{\textsf{Fortran Cheat Sheet}}}\\
  Zusammenstellung: Karl Hänsel, Stand: \today
  \section{Formalia laut FOPS}
  \begin{itemize}
    \item Allgemein: nur ASCII-Zeichen (keine Umlaut oder ß), nicht case sensitive, eine Zeile (höchstens 72 Zeichen) pro Befehl, Zeilenfortsetzung mit \lstinline|&|
    \item Namen allgemein: beginnen mit Buchstaben, es folgen Buchstaben, Ziffern oder Unterstriche (ohne Umlaute!), immer vor Deklaration \lstinline|implicit none| setzen und so implizite Typvereinbarung vermeiden, Konstanten/Parameter in GROSSBUCHSTABEN, Variablen in kleinbuchstaben
    \item Namensbeginn je nach Typ: logicals \lstinline|l<name>|, character \lstinline|y<name>|, loop indices \lstinline|j<name>|, local integers \lstinline|i<name>|, subprogram integer arguments \lstinline|k<name>|, local reals \lstinline|z<name>|, subprogram real arguments \lstinline|p<name>|, complex numbers \lstinline|c<name>|
    \item ACHTUNG: beim Einlesen von input sorgt Eingabe von \lstinline|/| für ein Ignorieren aller weiteren Eingaben in der Zeile, ggf. muss also \lstinline|'/'| genutzt werden
  \end{itemize}
  
  \section{Kompilieren}
  Einfaches Kompilieren: \lstinline[style=neutral]|gfortran -o <name>.x <name>.f90| \\
  Mehr Fehlermeldungen vorab (z.B. bei Division durch 0 oder Verwendung von Variablen vor Wertzuweisung): \lstinline[style=neutral]|gfortran -ffpe-trap=invalid,zero,overflow -finit-real=snan <name>.x <name>.f90|
  \section{Operatoren}
  Hierarchie der Operatoren:
  \begin{enumerate}
    \item Geklammerte Ausdrücke \lstinline|( )|
    \item Punktrechnung \lstinline|*, /|
    \item Strichrechnung \lstinline|+, -|
    \item Vergleichsoperatoren \lstinline|==, /=, >=, <, <=|
    \item Negation \lstinline|.not.|
    \item Konjunktion \lstinline|.and.|
    \item Disjunktion und Kontravalenz \lstinline|.or., .xor.|
    \item Äquivalenz und Antivalenz \lstinline|.eqv., .neqv.|
  \end{enumerate}
  außerdem: Verkettung mit \lstinline|//|\\
  Assoziativität der Operatoren: alle linksassoziativ, nur Potenz-Operator \lstinline|**| rechtsassoziativ\\
  Potenz mit \lstinline|**| ungenauer als reine Multiplikation\\
  CAVE: Aufgrund von Rundungsfehlern ist mathematische Assoziativität und Fortran-Assoziativität im Ergebnis nicht (immer) exakt gleich \\
  Werden Zahlen verschiedener Art verrechnet, wird die höhere Priorität übernommen (Hierarchie \lstinline|complex > real > integer|)\\
  Vergleichsoperatoren funktionieren für alle Zahlen und auch character-Variablen (Vergleich des entsprechenden ASCII-Codes), bei komplexen Zahlen nur \lstinline|==, /=|\\
  Zusätzliche (Sicherheits-)Klammern verlangsamen Programm nicht
  
  \section{Zahlen}
  
  \section{Kontrollstrukturen}
  \subsection{Verzweigungen}
  \begin{lstlisting}[caption={Einzeilige if-Anweisung}]
    if (<logical>) <einzelne Anweisung>
  \end{lstlisting}
  \begin{lstlisting}[caption={if-then}]
    if (<logical>) then
      <Anweisung(en)>
    endif
  \end{lstlisting}
  \begin{lstlisting}[caption={if-then-else}]
    if (<logical>) then
      <Anweisung(en) 1>
    else
      <Anweisung(en) 2>
    endif
  \end{lstlisting}
  \begin{lstlisting}[caption={if-then-else-if-...-else}]
    if (<logical 1>) then
      <Anweisung(en) 1>
    else if (<logical 2>) then
      <Anweisung(en) 2>
    else if (<logical 3>) then
      <Anweisung(en) 3>
    else
      <Anweisung(en) 4>
    endif
  \end{lstlisting}
  \subsection{Schleifen}
  
  \section{Formatierung und Strings}
  Strings verbinden: \lstinline|y_string = "String 1"   // "String 2"|
  \begin{lstlisting}[caption={String-Verkettung über Code-Zeilen hinweg},label=lst:stringkette]
    y_srting = "String 1" // &
               "String 2"
  \end{lstlisting}
  
  \section{Nützliche integrierte Funktionen}
  \lstinline|<integer> = nint(<real>)| rundet auf nächsten integer (im Gegensatz zu \lstinline|<integer> = <real>|, wo nur rationaler Teil abgeschnitten wird)
  
  
\end{document}
