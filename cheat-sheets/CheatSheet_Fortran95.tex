\documentclass[a4paper, twocolumn]{scrarticle}
\usepackage[left=1.5cm,right=1.5cm,top=2cm,bottom=2cm]{geometry}

% Standard-Schriftgröße auf footnoteisze setzen
\renewcommand{\normalsize}{\footnotesize}
\AtBeginDocument{\normalsize}
\renewcommand{\familydefault}{\sfdefault} % Serifenlose Schrift

\usepackage[utf8]{inputenc}
\usepackage[T1]{fontenc}
\usepackage[ngerman]{babel}

\usepackage[labelfont=bf]{caption} % Setzt Captions fett

\usepackage{hyperref}
\hypersetup{
  colorlinks=true,
  linkcolor=orange,
  citecolor=green,
  urlcolor=green
}

\setlength{\parindent}{0pt} % Kein Einrücken der Absätze
\setlength{\columnsep}{24pt} % Abstand zwischen den Spalten

% Syntax-Highlighting und Code-Umgebung
\usepackage{xcolor}
\usepackage{listings}
\lstdefinelanguage{Fortran95}{
  language=[95]Fortran,
  basicstyle=\ttfamily\small,
  keywordstyle=\color{blue},
  commentstyle=\color{gray},
  stringstyle=\color{purple},
  morekeywords={xor, kind, module, use, implicit, none, integer, real, double, precision, complex, logical, character, parameter, dimension, allocate, deallocate, contains, subroutine, function, end, if, then, else, endif, do, while, cycle, exit, select, case, default, write, read, print, format, open, close, rewind, backspace, inquire, stop, error, go, to, continue, return, call, data, save, common, equivalence, external, intrinsic, assign, pause, class, type, public, private, sequence, extends, import}
}

\lstset{
  language=Fortran95,
  basicstyle=\ttfamily\footnotesize,
  frame=single,
  xleftmargin=-12pt,
  framexleftmargin=-20pt,
  numbers=left,
  numberstyle=\tiny\color{gray},
  numbersep=-14pt,
  stepnumber=1,
  breaklines=true,
  tabsize=4,
  showstringspaces=false,
  captionpos=t,
  escapeinside={(*@}{@*)}  % Ermöglicht LaTeX-Befehle innerhalb von Listings
}

\lstdefinestyle{neutral}{
  language=bash,
  basicstyle=\ttfamily\small,
  numbers=left,
  xleftmargin=1pt,
  framexleftmargin=1pt,
  numberstyle=\tiny\color{gray},
  backgroundcolor=\color{white},
  frame=single,
  rulecolor=\color{black},
  captionpos=b,
  commentstyle=\color{black},
  stringstyle=\color{black},
  showstringspaces=false,
  keywordstyle=\color{black}, 
  morekeywords={}
}

\renewcommand{\lstlistingname}{Code}

% INLINE: \lstinline|<code>|
% UMGEBUNG: \begin{lstlisting}[caption={[<shortcaption>]<Caption>}, label=<label>] <codeblock> \end{lstlisting}
% LaTeX-Befehl innerhalb Umgebung: (*@<LaTeX-Syntax>@*)

% Cleveref
\usepackage{cleveref}
\crefname{listing}{Code-Beispiel}{Code-Beispiele}

\begin{document}
  
  {\Huge \textbf{\textsf{Fortran Cheat Sheet}}}
  Zusammenstellung: Karl Hänsel, Stand: \today
  \section{Formalia laut FOPS}
  \textbf{Allgemein:} nur ASCII-Zeichen (keine Umlaut oder ß), nicht case sensitive, eine Zeile (höchstens 72 Zeichen) pro Befehl, Zeilenfortsetzung mit \lstinline|&|\\
 \textbf{ Namen allgemein:} beginnen mit Buchstaben, es folgen Buchstaben, Ziffern oder Unterstriche (ohne Umlaute!), immer vor Deklaration \lstinline|implicit none| setzen und so implizite Typvereinbarung vermeiden, Konstanten/Parameter in GROSSBUCHSTABEN, Variablen in kleinbuchstaben\\
 \textbf{Namensbeginn je nach Typ:} logicals \lstinline|l<name>|, character \lstinline|y<name>|, loop indices \lstinline|j<name>|, local integers \lstinline|i<name>|, subprogram integer arguments \lstinline|k<name>|, local reals \lstinline|z<name>|, subprogram real arguments \lstinline|p<name>|, complex numbers \lstinline|c<name>|\\
 \textbf{CAVE:} beim Einlesen von input sorgt Eingabe von \lstinline|/| für ein Ignorieren aller weiteren Eingaben in der Zeile, ggf. muss also \lstinline|'/'| genutzt werden
  \section{Operatoren}
  \textbf{Deklaration von Variablen:} \lstinline|<vartype> :: <varname>|\\
  \textbf{Zuweisung von Variablen:} \lstinline|<varname> = <Wert>|\\
  \textbf{Hierarchie der Operatoren:}
  \begin{enumerate}
    \item Geklammerte Ausdrücke \lstinline|( )|
    \item Punktrechnung \lstinline|*, /|
    \item Strichrechnung \lstinline|+, -|
    \item Vergleichsoperatoren \lstinline|==, /=, >=, <, <=|
    \item Negation \lstinline|.not.|
    \item Konjunktion \lstinline|.and.|
    \item Disjunktion und Kontravalenz \lstinline|.or., .xor.|
    \item Äquivalenz und Antivalenz \lstinline|.eqv., .neqv.|
  \end{enumerate}
  \textbf{Verkettung:} mit \lstinline|//|\\
  \textbf{Assoziativität der Operatoren:} alle linksassoziativ, nur Potenz-Operator \lstinline|**| rechtsassoziativ\\
  \textbf{Potenz} mit \lstinline|**| ungenauer als reine Multiplikation\\
  \textbf{CAVE:} Aufgrund von Rundungsfehlern ist mathematische \textbf{Assoziativität} und Fortran-Assoziativität im Ergebnis nicht (immer) exakt gleich \\
  Werden \textbf{Zahlen verschiedener Art} verrechnet, wird die höhere Priorität übernommen, Hierarchie \lstinline|complex > real > integer|\\
  \textbf{Vergleichsoperatoren} funktionieren für alle Zahlen und auch character-Variablen (Vergleich des entsprechenden ASCII-Codes), bei komplexen Zahlen nur \lstinline|==, /=|\\
  \textbf{Zusätzliche (Sicherheits-)Klammern} verlangsamen Programm nicht
  
  \section{Datentypen}
  \subsection{Zahlen}
  \textbf{Ganze Zahlen:} \lstinline|integer| haben in Fortran 90/95 eine Tiefe von 32 Bit, also einen Wertebereich von $-2^{31}$ bis $2^{31}-1$, bzw. von $-2147483648$ bis $2147483647$\\
  \textbf{Umwandeln von \lstinline|integer| auf \lstinline|real|:} \lstinline|<real var> = real(<integer var>)|\\
  \textbf{Runden von \lstinline|real| auf nächsten \lstinline|integer|:} \lstinline|<integer> = nint(<real>)| (im Gegensatz zu \lstinline|<integer> = <real>|, wo nur rationaler Teil abgeschnitten wird)
  \subsection{Zeichenketten (\glqq Strings\grqq)}
  \textbf{ Strings verbinden:} \lstinline|y_string = "String 1"   // "String 2"|
  \begin{lstlisting}[caption={\bfseries String-Verkettung über Code-Zeilen hinweg},label=lst:stringkette]
    y_srting = "String 1" // &
    "String 2"
  \end{lstlisting}
  \textbf{Strings ändern:} Strings können wir arrays indiziert werden, sodass auch einzelne Zeichen gezielt geändert werden können: \lstinline|y<character var>(2:4) = "neu"|\\
  \textbf{Strings aus Wiederholung erstellen:} \lstinline|<character var> = repeat(<character>, <integer>)|, Anzahl muss $\geq 0$ sein\\
  \textbf{Leerzeichen am Schluss eines Strings entfernen:} \lstinline|<character var> = trim(<character var>)|\\
  
  \section{Kontrollstrukturen}
  \subsection{Verzweigungen}
  \begin{lstlisting}[caption={\bfseries Einzeilige if-Anweisung}]
    if (<logical>) <einzelne Anweisung>
  \end{lstlisting}
  \begin{lstlisting}[caption={if-then}]
    if (<logical>) then
      <Anweisung(en)>
    endif
  \end{lstlisting}
  \begin{lstlisting}[caption={\bfseries if-then-else}]
    if (<logical>) then
      <Anweisung(en) 1>
    else
      <Anweisung(en) 2>
    endif
  \end{lstlisting}
  \begin{lstlisting}[caption={\bfseries if-then-else-if-...-else}]
    if (<logical 1>) then
      <Anweisung(en) 1>
    else if (<logical 2>) then
      <Anweisung(en) 2>
    else if (<logical 3>) then
      <Anweisung(en) 3>
    else
      <Anweisung(en) 4>
    endif
  \end{lstlisting}
  \subsection{Schleifen}
  \begin{lstlisting}[caption={\bfseries do-Zählschleife (Schrittweite standardmäßig 1, kann auch negativ sein)}, label=lst:do-zahl]
    do j<varname> = i<startwert>, i<endwert> [, i<schrittweite>]
      <Anweisung(en)>
    enddo
  \end{lstlisting}
  \begin{lstlisting}[caption={\bfseries do-while-Schleife}, label=lst:do-while]
    do while (<logical>)
      <Anweisung(en)>
      ! Wiederholung bis <logical> .eqv. .false.
    enddo
  \end{lstlisting}
  \begin{lstlisting}[caption={\bfseries do-Schleife (CAVE!)}, label=lst:do]
    do
      <Anweisung(en)>
      ! Anweisungen werden immer wiederholt bis..
      if (<logical>) exit ! forciert Ende
    enddo
  \end{lstlisting}
 \textbf{Schleifensteuerung:} \lstinline|exit| sorgt für sofortiges Beenden der Schleife (sind Schleifen verschachtelt und nicht benannt, wird nur aktueller Schleifenkontext verlassen), \lstinline|cycle| sorgt für sofortigen Sprung zum Anfang der Schleife (für beide: egal, was danach noch im Code kommt)
  \begin{lstlisting}[caption={\bfseries Benannte do-Schleifen und explizites exit}]
    outer: do j_outer = 0, 5
      middle: do j_middle = 1, 4
        if (<logical>) exit outer
        inner: do j_inner = 3, 2, -1
          <Anweisung(en)>
        enddo inner
      enddo middle
    enddo outer
  \end{lstlisting}
 \textbf{Benennung von Schleifen nicht notwendig}, wenn kein explizites \lstinline|exit| benötigt wird, aber fördern Lesbarkeit\\
  \textbf{Faustregel für Schleifenauswahl:} Weiß man (konkret oder abstrakt aus vorherigen Anweisungen) wie oft genau etwas gemacht werden muss, dann wählt man eine do-Zählschleife (\cref{lst:do-zahl}), ist das nicht (genau) bekannt, eine do-while-Schleife (\cref{lst:do-while})
  
  \section{Felder/Arrays}
  Alle Werte in einem Feld haben den \textbf{gleichen Typ von Variable} \lstinline|<vartype>|\\
  Arrays haben einen (unveränderlichen) \textbf{rank} (Anzahl Dimensionen). Jede Dimension hat einen (unveränderlichen) shape \textbf{shape} (Anzahl Elemente pro Dimension) und einen (unveränderlichen) \textbf{extent} (Integer-Bereich, den die Indizes abdecken)\\
  Es kann \textbf{höchstens 7 Dimensionen} geben, die nicht alle den gleichen shape haben müssen. Wird nur ein shape \lstinline|n| angegeben, ist der \textbf{Start-Index standardmäßig 1} und der End-Index \lstinline|n|. Wird ein extent \lstinline|m:n| angegeben, so ist der shape automatisch gleich $n-m+1$.
  \begin{lstlisting}[caption={\bfseries Deklarationsmöglichkeiten arrays}]
    ! EINE DIMENSION
    ! nur shape=n setzen, Start-Index 1:
    <vartype>, dimension(<n>) :: <vektorname>
    ! extent mit eigenem Start-Index m:
    <vartype>, dimension(<m>:<n>) ::  <vektorname>
    ! ZWEI DIMENSIONEN
    ! nur shapes setzen:
    <vartype>, dimension(<shape 1>, <shape 2>) :: <matrixname>
    ! extents setzen:
    <vartype>, dimension(<m1>:<n1>, <m2>:<n2>) :: <natrixname>
    
    ! ALTERNATIV auch Deklaration rechts moeglich
    ! nur shape
    <vartype> :: <vektorname>(4)
    ! mit extent
    <vartype> :: <matrixname>(0:9, 0:9)
  \end{lstlisting}
  \begin{lstlisting}[caption={\bfseries Beispiele direkte Definition arrays}]
    ! Wochentage
    character(len=2), dimension(7) :: y_wochentage=["Mo", "Di", "Mi", "Do", "Fr", "Sa", "So"]
    ! geht auch mit \(...\):
    real, dimension(3) :: z_g=\(0, 0, -9.81\)
  \end{lstlisting}
  \textbf{Indizierung von arrays:} \lstinline|<arrayname>(x)| für Element \lstinline|x|, \lstinline|<arrayname>(r:s)| für Elemente \lstinline|r| bis \lstinline|s|, \lstinline|<arrayname>(:, x)| für die Elemente \lstinline|x| aus der gesamten ersten Dimension\\
  \textbf{Initialisierung von arrays:} kann im Deklarationsteil oder später erfolgen \lstinline|<integervektorname>(:) = 0| setzt alle Elemente auf 0 (\lstinline|<integervektorname> = 0| geht auch, schadet aber der Unterscheidbarkeit von dimensionslosen Variablen und arrays)\\
  Berechnungen zwischen arrays und Skalaren sowie mehreren arrays (wenn rank und shpaes übereinstimmen) sind möglich\\
  \textbf{WICHTIG:} \textbf{Index der ersten Dimension wird am schnellsten durchlaufen} und sollte daher in geschachtelten do-Zählschleifen \textbf{immer in der innersten Schleife} durchlaufen werden
  
  \section{Konstanten}
  Konstanten nutzen, sobald eine bestimmte Zahl mehrfach auftaucht (z.B. Dimensionen von Feldern in der Deklaration und in do-Zählschleife), um später leichter Programmanpassungen oder -erweiterungen an nur einem Ort vornehmen zu können\\
  \textbf{Definition von Konstanten direkt im Deklarationsteil:} \lstinline|<vartype>, parameter :: <CONSTNAME>=<constwert>| \\
 \textbf{Konstanten immer komplett in Großbuchstaben benennen}
 \section{Eingabe und Ausgabe}
 \subsection{Grundlagen}
 \textbf{Einlesen von Variablen über Terminal:} \lstinline|read(*,*) <var>|\\
 \textbf{Ausgabe von Variablen über Terminal:} \lstinline|write(*,*) <var>|\\
 \textbf{Standard-Argument:} \lstinline|(*,*)|\\
 \textbf{Datei lesen/schreiben:} \lstinline|(unit=<unit integer>, *)|\\
 \textbf{Formatierung beeinflussen:} \lstinline|(*,<format character>)|\\
 \textbf{Kombination:} möglich, aber \textbf{Einlesen von Dateien sollte in Standard-Formatierung erfolgen} (\lstinline|read(unit=<unit integer>, *)|), da Formatierung stark fehleranfällig beim Einlesen
 
 \subsection{Dateien lesen/schreiben}
 \subsubsection{Datei öffnen}
 \textbf{Grundlegendes Kommando:} \lstinline|open(unit=<integer>,file=<filename character>...)|\\
 \textbf{Unit:} nicht-negativer, zweistelliger \lstinline|integer| (einstellig können Peripheriegeräte sein)\\
 \textbf{File:} relativ wenn im gleichen Ordner, sonst Pfad mit angeben\\
 \textbf{Status:} \lstinline|status=<status character>| mit Möglichkeiten \lstinline|"old", "new", "unknwon" (default), "replace", "scratch"| (scratch: nur temporär während Programmlauf existent)\\
 \textbf{Action:} \lstinline|action=<action character>| mit den Möglichkeiten \lstinline|"read", "write", "readwrite" (default)|\\
 \textbf{Fehlervariable:} \lstinline|iostat=<integer var>|, ist \lstinline|<integer var>|$\neq 0$, gab es Fehler, z.B. bei EOF (End of File), \textbf{Wichtig: immer nutzten!}\\
 \textbf{Position:} \lstinline|position="append"|, wenn am Ende angefügt statt von Anfang überschrieben \lstinline|(default)| werden soll\\
 \textbf{Format:} \lstinline|form=<form character>| mit den Möglichkeiten \lstinline|"formatted" (default), "unformatted"|
 \subsubsection{Weitere Datei-Anweisungen}
 \textbf{Datei schließen:} \lstinline|close(unit=<unit integer>)|, eigentlich automatisch am Programmende, insbesondere bei großen Datenmengen aber sinnvoll, sobald nicht mehr benötigt\\
 \textbf{Existenzprüfung:} \lstinline|inquire(file=<filename character>, exist=<exist logical>)| gibt an, ob eine Datei existiert\\
 \textbf{Zurücksetzen:} \lstinline|rewind(unit=<unit integer>)| setzt die Position der Datei wieder auf Anfang zum erneuten Lesen/Schreiben\\
 \textbf{Eine Zeile zurück:} \lstinline|backspace(unit=<unit integer>)| setzt Position der Datei um einen record (=Zeile) zurück
 
 \subsection{Formatierung}
 \textbf{Grundform Format-String:} \lstinline|"(<format>)"|\\
 \textbf{Mögliche Zahlenangaben:}  Datenfeldweite (muss groß genug sein, sonst werden Sterne statt Zahl ausgegeben, unten \lstinline|w|), Anzahl Dezimalstellen (unten \lstinline|d|), Anzahl Ziffern im Exponenten (unten \lstinline|e|), Anzahl ausgegebener Ziffern (ggf. führende Nullen hinzugefügt, unten \lstinline|m|)\\
 \textbf{\lstinline|integer|:} \lstinline|Iw[.m]|\\
 \textbf{\lstinline|real|:} Fixpunktdarstellung: \lstinline|Fw.d| (\lstinline|w|$\geq$\lstinline|d|$+2$), Exponentialdarstellung: \lstinline|Ew.d| (\lstinline|w|$\geq$\lstinline|d|$+7$), Exponentialdarstellung mit fester Ziffernzahl im Exponenten: \lstinline|Ew.dEe| (\lstinline|w|$\geq$\lstinline|d|$+$\lstinline|e|$+5$), Exponentialdarstellung mit fester Zahl an \lstinline|n| Stellen vor dem Komma: \lstinline|nPEw.d|, automatische Wahl von \lstinline|F| oder \lstinline|E| Format: \lstinline|Gw.d|\\
 \textbf{\lstinline|character|:} \lstinline|A[w]|\\
 \textbf{\lstinline|logical|:} \lstinline|Lw|\\
 \textbf{Positionierung in \lstinline|n|ter Spalte (Tabulator):} \lstinline|Tn|\\
 \textbf{Ausgabe von \lstinline|n| Leerstellen:} \lstinline|nX|\\
 \textbf{Zeilensprung:} \lstinline|/|\\
 \textbf{Zeilenvorschub unterdrücken:} \lstinline|"(<format>$)"|\\
 \textbf{\lstinline|n|-fach wiederholen:} \lstinline|n(<format>)|, bzw. \lstinline|n(<character>)|\\
 \textbf{Auslagerung in Variable:} sinnvoll, wenn mehrfach verwendet oder gar zwischendurch verändert werden soll
  
  \section{Kompilieren}
  \textbf{Einfach:} \lstinline[style=neutral]|gfortran -o <name>.x <name>.f90| \\
  \textbf{Mehr Fehlermeldungen vorab} (z.B. bei Division durch 0 oder Verwendung von Variablen vor Wertzuweisung): \lstinline[style=neutral]|gfortran -ffpe-trap=invalid,zero,overflow -finit-real=snan <name>.x <name>.f90|
\end{document}
