% Dokumenten-Einstellungen
\documentclass[a4paper, twocolumn]{scrarticle}
\usepackage[left=1.5cm,right=1.5cm,top=1.5cm,bottom=1.5cm]{geometry}

% Spracheinstellungen
\usepackage[utf8]{inputenc}
\usepackage[T1]{fontenc}
\usepackage[ngerman]{babel}

% Standard-Schrift auf footnoteisze  und serifenlos setzen
\renewcommand{\normalsize}{\footnotesize}
\AtBeginDocument{\normalsize}
\renewcommand{\familydefault}{\sfdefault}

% Caption-Nummerierung soll fettgedruckt sein
\usepackage[labelfont=bf]{caption}

\setlength{\parindent}{0pt} % Kein Einruecken beim Absatz
\setlength{\columnsep}{24pt} % Abstand zwischen den Spalten

% fuer ein bisschen Mathe
\usepackage{amsmath}

% Syntax-Highlighting und Code-Umgebung
\usepackage{xcolor}
\usepackage{listings}
\lstdefinelanguage{Fortran95}{
  basicstyle=\ttfamily\small,
  commentstyle=\color{gray},
  stringstyle=\color{olive},
  moredelim=[s][\color{purple}]{<}{>},  % Platzhalter in spitzen Klammern faerben
  morekeywords={kind, module, use, allocate, deallocate, contains, subroutine, function, end, if, then, else, endif, do, enddo, while, cycle, exit, select, case, default, print, stop, error, go, to, continue, return, call, data, save, common, equivalence, external, intrinsic, assign, pause, class, type, public, private, sequence, extends, import},
  morekeywords=[2]{not, and, or, xor, eqv, neqv, true, false}, % Logik-Operatoren
  morekeywords=[3]{write, read, close, rewind, backspace, inquire, mod, nint, abs, repeat, trim},  % built-in Prozeduren/Funktionen
  morekeywords=[4]{format, kind, len, unit, status, iostat, action, postion, form, exist, file, intent}, % Standard-Argumente
  morekeywords=[5]{integer, real, double, precision, complex, logical, character, implicit, none, parameter, dimension, },  % Datentypen etc
  keywordstyle=\color{blue},
  morekeywords=[6]{in, out, inout},
  keywordstyle=[2]\color{cyan},
  keywordstyle=[3]\color{brown},
  keywordstyle=[4]\color{orange},
  keywordstyle=[5]\color{violet},
  keywordstyle=[6]\color{green}
}
\lstset{
  language=Fortran95,
  basicstyle=\ttfamily\footnotesize,
  frame=single,
  xleftmargin=-6pt,
  framexleftmargin=-8pt,
  numbers=left,
  numberstyle=\tiny\color{gray},
  numbersep=-2pt,
  stepnumber=1,
  breaklines=true,
  tabsize=4,
  showstringspaces=false,
  captionpos=t
}
\lstdefinestyle{neutral}{
  % lstset wird übernommen
  language=bash,
  commentstyle=\color{black},
  stringstyle=\color{black},
  keywordstyle=\color{black}, 
  morekeywords={}
}

% Listing soll in Caption als 'Code' bezeichnet werden
\renewcommand{\lstlistingname}{Code}

% INLINE: \lstinline|<code>|
% UMGEBUNG: \begin{lstlisting}[caption={[<shortcaption>]<Caption>}, label=<label>] <codeblock> \end{lstlisting}

\begin{document}

% Überschrift
{\Huge \textbf{\textsf{Fortran Cheat Sheet}}}
Zusammenstellung: Karl Hänsel, Stand: \today

\section{Formalia laut FOPS}
\textbf{Allgemein:} nur ASCII-Zeichen (keine Umlaute oder ß), Groß- oder Kleinschreibung Fortran egal, eine Zeile (höchstens 72 Zeichen) pro Befehl, Zeilenfortsetzung mit \lstinline|&|\\
\textbf{ Namen allgemein:} beginnen mit Buchstaben, es folgen Buchstaben, Ziffern oder Unterstriche, immer \lstinline|implicit none| vor Deklaration setzen und so implizite Typvereinbarung vermeiden, Konstanten/Parameter in GROSSBUCHSTABEN, Variablen in kleinbuchstaben\\
\textbf{Namensbeginn je nach Typ:} logicals \lstinline|l|, character \lstinline|y|, loop indices \lstinline|j|, local integers \lstinline|i|, subprogram integer arguments \lstinline|k|, local reals \lstinline|z|, subprogram real arguments \lstinline|p|, complex numbers \lstinline|c|\\
\textbf{CAVE:} beim Einlesen von input sorgt Eingabe von \lstinline|/| für ein Ignorieren aller weiteren Eingaben in der Zeile, ggf. muss also \lstinline|'/'| genutzt werden

\section{Operatoren}
\textbf{Deklaration von Variablen:} \lstinline|<type>| \lstinline|:: <varname>|\\
\textbf{Zuweisung von Variablen:} \lstinline|<varname>| \lstinline|= <Wert>|\\
\textbf{Hierarchie der Operatoren:}
\begin{enumerate}
  \item Geklammerte Ausdrücke \lstinline|( )|
  \item Punktrechnung \lstinline|*, /|
  \item Strichrechnung \lstinline|+, -|
  \item Vergleichsoperatoren \lstinline[style=neutral]|==, /=, >, >=, <, <=|
  \item Negation \lstinline|.not.|
  \item Konjunktion \lstinline|.and.|
  \item Disjunktion und Kontravalenz \lstinline|.or., .xor.|
  \item Äquivalenz und Antivalenz \lstinline|.eqv., .neqv.|
\end{enumerate}
\textbf{Verkettung:} mit \lstinline|//|\\
\textbf{Assoziativität der Operatoren:} alle linksassoziativ, nur Potenz-Operator \lstinline|**| rechtsassoziativ, \textbf{CAVE:} wegen Rundungsfehlern nicht immer exakt gleich zur mathematischen Assoziativität\\
\textbf{Potenz} mit \lstinline|**| ungenauer als reine Multiplikation\\
Werden \textbf{Zahlen verschiedener Art} verrechnet, wird die höhere Priorität übernommen, Hierarchie \lstinline|complex > real > integer|\\
\textbf{Vergleichsoperatoren} funktionieren für alle Zahlen und auch character-Variablen (Vergleich des ASCII-Codes), bei komplexen Zahlen nur \lstinline|==, /=|\\
\textbf{Zusätzliche (Sicherheits-)Klammern} verlangsamen nicht

\section{Zahlen}
\textbf{Ganze Zahlen:} \lstinline|integer| standardmäßig \lstinline|integer(kind=4)| mit Tiefe von 32 Bit, also Wertebereich von $-2^{31}$ bis $2^{31}-1$, bzw. von $-2147483648$ bis $2147483647$, 64 Bit mit \lstinline|integer(kind=8)| (möglich sind 1, 2, 4 und 8)\\
\textbf{CAVE:} \lstinline|<integer>| \lstinline|= 5.7| liefert keine Fehlermeldung, schneidet aber am Punkt ab!\\
\textbf{Modulo:} \lstinline|mod(<number 1>, <number 2>)| liefert \textbf{Rest der Division} (Vorzeichen entspricht dem von \lstinline|<number 1>|) \\
\textbf{Umwandeln von \lstinline|integer| auf \lstinline|real|:} \lstinline|<real>| \lstinline|= real(<integer>)|\\
\textbf{Runden von \lstinline|real| auf nächsten \lstinline|integer|:} \lstinline|<integer>| \lstinline|= nint(<real>)| (im Gegensatz zu \lstinline|<integer>| \lstinline|= <real>|, wo nur rationaler Teil abgeschnitten wird)\\
\textbf{Faustregel zur Wahl des Zahlentyps:} \lstinline|integer| nur, wenn es um Anzahlen oder Indizes geht; soll gerechnet werden, dann als \lstinline|real|\\
\textbf{Komplexe Zahlen:} nach \lstinline|complex :: c<varname>| Zuweisung z.B. mit \lstinline|c<varname>=(1.0,1.0)| (hier $1+\textbf{i}$)\\
\textbf{Absolutbetrag einer Zahl:} \lstinline|abs(<number>)|
\begin{lstlisting}[caption={\bfseries Beispiel: CAVE - implizite Umwandlungen bei Division}]
  ! BEISPIEL
  real    :: z
  integer :: i
  i = 57
  i = i/10  ! 5
  i = i/10.  ! 5, obwohl i/10.==5.7
  z = i/10  ! 5.0, da i/10 == 5
  z = i/10.  ! 5.7
\end{lstlisting}

\section{Zeichenketten (\glqq Strings\grqq)}
\textbf{Strings:} in Fortran mit festzulegender Länge an Zeichen  (höchstens 256) \lstinline|character(len=<integer>) :: <character>|\\
\textbf{String-Verkettung:} \lstinline|y_string = "String 1"   // "String 2"|
\begin{lstlisting}[caption={\bfseries Beispiel: String-Verkettung über Code-Zeilen hinweg},label=lst:stringkette]
  y_string = "String 1" // &
  "String 2"
\end{lstlisting}
\textbf{Strings ändern:} Strings können indiziert werden, sodass auch einzelne Zeichen gezielt geändert werden können: \lstinline|<character>(2:4) = "neu"|\\
\textbf{Strings aus Wiederholung erstellen:} \lstinline|<character>| \lstinline|= repeat(<character>, <integer>)|, Anzahl muss $\geq 0$ sein\\
\textbf{Leerzeichen am Schluss eines Strings entfernen:} \lstinline|<character>| \lstinline|= trim(<character>)|

\section{Konstanten}
\textbf{Konstanten:} nutzen, sobald ein Wert mehrfach auftaucht (z.B. Dimensionen von Feldern in der Deklaration und in do-Zählschleife), um später leichter Programmanpassungen oder -erweiterungen an nur einem Ort vornehmen zu können\\
\textbf{Definition von Konstanten direkt im Deklarationsteil möglich:} \lstinline|<type>, parameter :: <CONSTNAME>=<Wert>| \\
\textbf{Konstanten immer komplett in Großbuchstaben benennen}

\section{Verzweigungen (if-Strukturen)}
\begin{lstlisting}[caption={\bfseries Verzweigungen mit if}]
  ! einzeilige if-Anweisung
  if (<logical>) <einzelne Anweisung>
  
  ! if-then
  if (<logical>) then
    <Anweisung(en)>
  endif
  
  ! if-then-else
  if (<logical>) then
    <Anweisung(en) 1>
  else
    <Anweisung(en) 2>
  endif
  
  ! if-then-else if-...-else
  if (<logical 1>) then
    <Anweisung(en) 1>
  else if (<logical 2>) then
    <Anweisung(en) 2>
  else if (<logical 3>) then
    <Anweisung(en) 3>
  else
    <Anweisung(en) 4>
  endif
\end{lstlisting}

\section{Schleifen}
\begin{lstlisting}[caption={\bfseries do-Schleifen}]
  ! do-Zaehlschleife (Schrittweite Standard 1, kann auch negativ sein)
  do <loop integer> = <start integer>, <end integer> [, <step integer>]
    <Anweisung(en)>
  enddo
  
  ! do-while-Schleife
  do while (<logical>)
    <Anweisung(en)>
    ! Wiederholung bis <logical> .eqv. .false.
  enddo
  
  ! do-Schleife (CAVE!)
  do
    <Anweisung(en)>
    ! Anweisungen werden immer wiederholt bis..
    if (<logical>) exit ! forciert Ende
  enddo
  
  ! implizite do-Schleife, Beispiel
  write(*,*) i_array(j_index), j_index=1, I_MAX
\end{lstlisting}
\textbf{Schleifensteuerung:} \lstinline|exit| sorgt für sofortiges Beenden der Schleife (sind Schleifen verschachtelt und nicht benannt, wird nur aktueller Schleifenkontext verlassen), \lstinline|cycle| sorgt für sofortigen Sprung zum Anfang der Schleife 
\begin{lstlisting}[caption={\bfseries Beispiel: Benannte do-Schleifen und explizites exit}]
  outer: do j_outer = 0, 5
    middle: do j_middle = 1, 4
      if (<logical>) exit outer
      inner: do j_inner = 3, 2, -1
        <Anweisung(en)>
      enddo inner
    enddo middle
  enddo outer
\end{lstlisting}
\textbf{Benennung von Schleifen nicht notwendig}, wenn kein explizites \lstinline|exit| benötigt wird, aber fördert Lesbarkeit\\
\textbf{Faustregel für Schleifenauswahl:} weiß man (konkret oder abstrakt aus vorherigem Programmverlauf) wie oft genau etwas gemacht werden muss: do-Zählschleife, ist das nicht (genau) bekannt: do-while-Schleife.

\section{Felder/Arrays}
\textbf{Variablentyp:} für alle Werte in einem Feld  gleich\\
\textbf{Arrays} haben einen (unveränderlichen) \textbf{rank} (Anzahl Dimensionen). Jede Dimension hat einen (unveränderlichen) shape \textbf{shape} (Anzahl Elemente pro Dimension) und einen (unveränderlichen) \textbf{extent} (Integer-Bereich, den die Indizes abdecken)\\
\textbf{höchstens 7 Dimensionen}, nicht unbedingtgleicher shape\\
\textbf{Nur shape \lstinline|n| angeben:} extent automatisch \lstinline|1:n|\\
\textbf{Nur extent \lstinline|m:n| angegeben:} shape automatisch $n-m+1$.
\begin{lstlisting}[caption={\bfseries Deklarationsmöglichkeiten für arrays}]
  ! EINE DIMENSION (Vektor)
  ! nur shape=n setzen, Start-Index 1:
  <type>, dimension(n) :: <varname>
  ! extent mit eigenem Start-Index m:
  <type>, dimension(m:n) :: <varname>
  
  ! ZWEI DIMENSIONEN (2D-Matrix)
  ! nur shapes n1, n2 setzen:
  <type>, dimension(n1, n2) :: <varname>
  ! extents setzen:
  <type>, dimension(m1:n1, m2:n2) :: <varname>
  
  ! ALTERNATIV auch Deklaration rechts moeglich
  ! nur shape
  <type> :: <varname>(n)
  ! mit extent
  <type> :: <varname>(m1:n1, m2:n2)
\end{lstlisting}
\begin{lstlisting}[caption={\bfseries Beispiele direkte Definition arrays}]
  ! Wochentage
  character(len=2), dimension(7) :: y_wochentage=["Mo", "Di", "Mi", "Do", "Fr", "Sa", "So"]
  ! geht auch mit \(...\):
  real, dimension(3) :: z_g=\(0, 0, -9.81\)
\end{lstlisting}
\textbf{Indizierung von arrays:} \lstinline|<arrayname>(x)| für Element \lstinline|x|, \lstinline|<arrayname>(r:s)| für Elemente \lstinline|r| bis \lstinline|s|, \lstinline|<arrayname>(:, x)| für die Elemente \lstinline|x| aus der gesamten ersten Dimension\\
\textbf{Initialisierung von arrays:} kann im Deklarationsteil oder später erfolgen \lstinline|<arrayname>(:) = 0| setzt alle Elemente auf 0 (\lstinline|<arrayname>| \lstinline|= 0| geht auch, schadet aber der Unterscheidbarkeit von dimensionslosen Variablen und arrays)\\
Berechnungen zwischen arrays und Skalaren sowie mehreren arrays (wenn rank und shpaes übereinstimmen) sind möglich\\
\textbf{CAVE:} \textbf{Index der ersten Dimension wird am schnellsten durchlaufen} und sollte daher in geschachtelten do-Zählschleifen \textbf{immer in der innersten Schleife} durchlaufen werden\\
\textbf{Standard-Funktionen für Felder:} \lstinline|dot_product(<array1>, <array2>)| für Skalarprodukt, \lstinline|maxval(<array>), minval(<array>)| geben Minimum und Maximum für Zahlen zurück, \listline|maxloc(<array>), minloc(<array>)| die entsprechenden Indizes als Felder, \lstinline|sum(<array>), size(<array>)| geben Summe und Anzahl aller Feldwerte zurück\\
\textbf{Arrays zuweisen (eindimensional):} \lstinline|<array>(:) = [<value1>, <value2> ...]| (alt \lstinline|(/.../)|), Zuweisung über implizites \lstinline|do| mit \lstinline|<array>(:) = [ (j, j=<integer>, <integer>) ]|\\
\textbf{Arrays zuweisen (mehrdimensional):} Standardwert für alle Felder \lstinline|<array>(:,:,:,...) = <value>|, Nutzung der Standard-Funktion \lstinline|reshape| unter Angabe der Form, z.B. für 2 x 2 Matrix \lstinline|<array>(2,2) = reshape([<value11>, <value12>, <value21>, <value22>], [2,2])|
\begin{lstlisting}[caption={\bfseries Allocatable Arrays (Genaue Größe nicht bei Deklaration bekannt)}]
  implicit none
  ! Anzahl Dimensionen muss bekannt sein, deren Umfang aber nicht
  real, allocatable, dimension(:,:) :: z_array
  integer :: i_alloc_stat ! optional
  
  ! genaue Deklaration dann im Programm (optional mit Fehlervariable zur Abfrage hinterher)
  allocate(z_array(<integer1>, <integer2>), stat=i_alloc_stat)
  if (i_alloc_stat /= 0) write(*,*) "Fehler, Speicherplatz reicht nicht aus!"
  ! Abfrage, ob bereits Speicherplatz zugewiesen wurde
  if (allocated(z_array)) write(*,*) "erfolgreich!"
  ! optional: Speicherplatz freigeben
  deallocate(z_array)
\end{lstlisting}

\section{Ein- und Ausgabe}
\subsection{Grundlagen}
\textbf{Einlesen von Variablen über Terminal:} \lstinline|read(*,*) <varname>|\\
\textbf{Ausgabe von Variablen über Terminal:} \lstinline|write(*,*) <Wert>|\\
\textbf{Datei lesen/schreiben:} \lstinline|(unit=<integer>, *)|\\
\textbf{Formatierung beeinflussen:} \lstinline|(*,<character>)|\\
\textbf{Kombination:} möglich, aber \textbf{Einlesen von Dateien sollte in Standard-Formatierung erfolgen} (\lstinline|read(unit=<integer>, *)|), da Formatierung stark fehleranfällig beim Einlesen

\subsection{Dateien lesen/schreiben}
\textbf{Grundlegendes Kommando (Pflichtargumente):} \lstinline|open(unit=<integer>,file=<character>)|\\
\textbf{\lstinline|unit|:} nicht-negativer, zweistelliger \lstinline|integer| (einstellig könnten Peripheriegeräte sein)\\
\textbf{\lstinline|file|:} Dateiname wenn im gleichen Ordner, sonst mit Pfad\\
\textbf{Status:} \lstinline|status=<status>| mit Möglichkeiten \lstinline|"old", "new", "unknwon" (default), "replace", "scratch"| (scratch: nur temporär während Programmlauf existent)\\
\textbf{Aktion:} \lstinline|action=<action>| mit den Möglichkeiten \lstinline|"read", "write", "readwrite" (default)|\\
\textbf{Fehlervariable:} \lstinline|iostat=<integername>|, ist \lstinline|<integername>| $\neq 0$, gab es Fehler, z.B. EOF (End of File), \textbf{immer prüfen!}\\
\textbf{Position:} \lstinline|position="append"|, wenn am Ende angefügt statt von Anfang überschrieben \lstinline|(default)| werden soll\\
\textbf{Format:} ohne Format mit \lstinline|form="unformatted"|\\
\textbf{Datei schließen:} \lstinline|close(unit=<integer>)|, eigentlich automatisch am Programmende, insbesondere bei großen Datenmengen aber sinnvoll, sobald nicht mehr benötigt\\
\textbf{Existenzprüfung:} gibt an, ob eine Datei existiert \lstinline|inquire(file=<character>, exist=<logicalname)| \\
\textbf{Zurücksetzen:} \lstinline|rewind(unit=<integer>)| setzt die Position der Datei wieder auf Anfang zum erneuten Lesen/Schreiben\\
\textbf{Eine Zeile zurück:} \lstinline|backspace(unit=<integer>)| setzt Position der Datei um einen record (=Zeile) zurück

\subsection{Formatierung}
\textbf{Grundform Format-String:} \lstinline|"(<character>)"|\\
\textbf{Mögliche Zahlenangaben:}  Datenfeldweite (muss groß genug sein, Dezimalpunkt zählt mit, unten \lstinline|w|), Anzahl Dezimalstellen (unten \lstinline|d|), Anzahl Ziffern im Exponenten (unten \lstinline|e|), Anzahl ausgegebener Ziffern (ggf. führende Nullen hinzugefügt, unten \lstinline|m|)\\
\textbf{\lstinline|integer|:} \lstinline|Iw[.m]|\\
\textbf{\lstinline|real|:} Fixpunktdarstellung: \lstinline|Fw.d| (wobei \lstinline|w|$\geq$\lstinline|d|$+2$), Exponentialdarstellung: \lstinline|Ew.d| (wobei \lstinline|w|$\geq$\lstinline|d|$+7$), Exponentialdarstellung mit fester Exponentenlänge: \lstinline|Ew.dEe| (wobei \lstinline|w|$\geq$\lstinline|d|$+$\lstinline|e|$+5$), Exponentialdarstellung mit fester Stellenzahl \lstinline|n| vor dem Komma: \lstinline|nPEw.d|, automatische Wahl von \lstinline|F| oder \lstinline|E| Format: \lstinline|Gw.d|\\
\textbf{\lstinline|character|:} \lstinline|A[w]|\\
\textbf{\lstinline|logical|:} \lstinline|Lw|\\
\textbf{Positionierung in \lstinline|n|ter Spalte (Tabulator):} \lstinline|Tn|\\
\textbf{Ausgabe von \lstinline|n| Leerstellen:} \lstinline|nX|\\
\textbf{Zeilensprung:} \lstinline|/|\\
\textbf{Zeilenvorschub unterdrücken:} \lstinline|"(<character>$)"|\\
\textbf{\lstinline|n|-fach wiederholen:} \lstinline|n(<character>)|\\
\textbf{Auslagerung in Variable:} sinnvoll, wenn mehrfach verwendet oder  zwischendurch verändert werden soll
\textbf{Standard-Funktionen für Text-Formatierung:} \lstinline|trim(<character>)| entfernt Leerzeichen am Ende, \lstinline|adjustl(<character>), adjustr(<character>)| richten Text links-, bzw. rechtsbündig aus, \lstlinline|len(<character>), len_trim(<character>)| geben Länge zurück (ggf. ohne Leerzeichen am Ende), \lstinline|index(<character1>, <character2>)| gibt Position zurück, an der Muster \lstinline|<character2>| in \lstinlne|<character1>| auftaucht 

\subsection{Parameterübergabe bei Programmaufruf}
\textbf{Anzahl der übergebenen Kommandozeilenparameter abfragen:} \lstinline|<integer> = command_argument_count()|\\
\textbf{Parameter auf Variablen einlesen:} \lstinline|call get_command_argument(<integer>, <charactervariable>)| (dabei \lstinline|<integer>|\overset{!}{\leq}\lstinline|commandargument_count()|), immer als \lstinline|character| eingelesen, Umwandlung z.B. über \lstinline|read(<charactervariable>,*) <numericalvariable>|
\textbf{Programmaufruf:} z.B. für zwei \lstinline|integer|-Argumente \lstinline[bash]|<programname>.x <integer1> <integer2>|

\section{Unterprogramme und Module}
\textbf{Speichern:} entweder alles in einer Datei mit Hauptprogramm oder einzelne Dateien möglich\\
\textbf{Subroutinen:} \lstinline|subroutine| führt Code aus, nutzt dazu die als Argumente übergebenen Werte, Aufruf mit \lstinline|call <subroutine>(<Argument(e)>)|\\
\textbf{Funktionen:} \lstinline|function| gibt einen Wert zurück, nutzt dazu die als Argumente übergebenen Werte, Zuweisung mit \lstinline|<varname>| \lstinline|= <function>(<Argument(e)>)|, muss im Hauptprogramm vorher mit entsprechendem Datentyp definiert werden: \lstinline|<type>| \lstinline|:: <function>|\\
\textbf{Deklarationen in Unterprogrammen:}  angeben, ob ein Argument nur eingelesen und nicht verändert (\lstinline|intent(in)|), unabhängig vom Originalwert hinterher zurückgegeben (\lstinline|intent(out)|) oder gelesen UND verändert werden soll (\lstinline|intent(inout)|)\\
\textbf{CAVE:} \textbf{Anzahl und Reihenfolge der Argumente} müssen zwischen Deklaration und Aufruf übereinstimmen!\\
\textbf{Variable Längendeklaration:} ist z.B. String-Länge  eines einzugebenden Arguments in Unterprogramm nicht bekannt, kann sie mit \lstinline|character(len=*) :: <Argument>| deklariert werden\\
\textbf{Modul:} enthält nicht ausführbaren  Programminhalt (Deklarationen, Wertzuweisungen, ...), der von allen Programmteilen genutzt wird, bei Einzelkompilierung  Mdul zuerst kompilieren\\
\textbf{Einbindung Modul in (Unter-)Programme:} \lstinline|use <modulename>|\\
\textbf{CAVE:} alle Unterprogramme haben auf den \textbf{aktuellen Zustand einer Modul-Variable} im Programmverlauf Zugriff (auch wenn gemeinsames Modul eigentlich initialisiert)\\
\textbf{Lokale Variablen:} wird eine Variable nicht über ein Modul \glqq verbreitet\grqq, gilt sie immer nur lokal in der jeweiligen Programmeinheit (Programm, Subroutine, Funktion)
\begin{lstlisting}[caption={\bfseries Dateischema Modul, Haupt- und Unterprogramme }]
  module <modulename>
  ! erklaerender Kommentar
  implicit none
  <globale Deklarationen, Definitionen, Initialisierungen>
  end module <modulename>
  
  program <progname>
  ! erklaerender Kommentar
  use <modulename>
  implicit none
  <lokale Deklarationen (auch zu nutzende Funktionen), Definitionen, Initialisierungen>
  <Anweisungsteil>
  stop
  end program <progname>
  
  <type> function <funcname>(<Arugment(e)>)
  ! erklaerender Kommentar
  use <modulename>
  implicit none
  <type>, intent(<intent>) :: <argname>
  <weitere lokale Deklarationen, Definitionen, Initialisierungen>
  <Anweisungen>
  return
  end function <funcname>
  
  subroutine <subname>(<Arguemt(e)>)
  ! erklaerender Kommentar
  use <modulename>
  implicit none
  <type>, intent(<intent>) :: <argname>
  <weitere lokale Deklarationen, Definitionen, Initialisierungen>
  <Anweisungen>
  return
  end subroutine <subname>
\end{lstlisting}

\section{Kompilieren}
\textbf{Einfach:} \lstinline[style=neutral]|gfortran -o <progname>.x <progname>.f90| \\
\textbf{Mehr Fehlermeldungen vorab} (z.B. bei Division durch 0 oder Verwendung von Variablen vor Wertzuweisung): \lstinline[style=neutral]|gfortran -ffpe-trap=invalid,zero,overflow -finit-real=snan <progname>.x <progname>.f90|\\
\textbf{Bei Unterprogrammnutzung:} \lstinline[style=neutral]|gfortran -o <progname>.x <progname>.f90 <subprogname>.f90| oder alternativ:
\begin{lstlisting}[style=neutral,caption={\textbf{Kompilieren von Haupt- und Unterprogrammenn}}]
  gfortran -c <modulename>.f90  ! zuerst!
  gfortran -c <progname>.f90 
  gfortran -c <subprogname>.f90 
  gfortran -o <progname>.x <programe>.o <subprogname>.o <modulename>.o
\end{lstlisting}
\textbf{Temporäre Dateien:} die bei Einzelkompilierung von Unterprogrammen und Modulen erzeugten Dateien werden nach Kompilierung nicht mehr benötigt und können gelöscht werden
\end{document}
