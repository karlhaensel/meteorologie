% Dokumenten-Einstellungen
\documentclass[a4paper, twocolumn]{scrarticle}
\usepackage[left=1.5cm,right=1.5cm,top=1.5cm,bottom=1.5cm]{geometry}

% Spracheinstellungen
\usepackage[utf8]{inputenc}
\usepackage[T1]{fontenc}
\usepackage[ngerman]{babel}

% Standard-Schrift auf footnoteisze  und serifenlos setzen
\renewcommand{\normalsize}{\footnotesize}
\AtBeginDocument{\normalsize}
\renewcommand{\familydefault}{\sfdefault}

% Caption-Nummerierung soll fettgedruckt sein
\usepackage[labelfont=bf]{caption}

\setlength{\parindent}{0pt} % Kein Einruecken beim Absatz
\setlength{\columnsep}{24pt} % Abstand zwischen den Spalten

% fuer ein bisschen Mathe
\usepackage{amsmath}

% Syntax-Highlighting und Code-Umgebung
\usepackage{xcolor}
\usepackage{listings}
\lstdefinelanguage{Fortran95Konkret}{ % keine <...> Platzhalter
  basicstyle=\ttfamily\small,
  commentstyle=\color{gray},
  morecomment=[l]{!},
  stringstyle=\color{olive},
  morekeywords={module, allocate, deallocate, subroutine, function, end, if, then, else, endif, do, enddo, while, cycle, exit, select, case, default, print, stop, error, go, to, continue, return, call, data, save, common, equivalence, external, intrinsic, assign, pause, class, type, public, private, sequence, extends, import, where, elsewhere, endwhere, forall, interface, operator, procedure, program},
  morekeywords=[2]{not, and, or, xor, eqv, neqv, true, false}, % Logik-Operatoren
  morekeywords=[3]{write, read, close, rewind, backspace, inquire, mod, nint, abs, repeat, trim, adjustl, adjustr, len_trim, index, command_argument_count, get_command_argument, reshape, dot_product, maxval, minval, maxloc, minloc, sum, size},  % built-in Prozeduren/Funktionen
  morekeywords=[4]{format, kind, len, unit, status, iostat, action, postion, form, exist, file, intent, position, nml, recl, rec, access, kind}, % Standard-Argumente
  morekeywords=[5]{integer, real, double, precision, complex, logical, character, implicit, none, parameter, dimension, namelist, contains, allocatable, optional, use},  % Datentypen, Deklaration etc
  keywordstyle=\color{blue},
  morekeywords=[6]{in, out, inout},
  keywordstyle=[2]\color{cyan},
  keywordstyle=[3]\color{brown},
  keywordstyle=[4]\color{orange},
  keywordstyle=[5]\color{violet},
  keywordstyle=[6]\color{green}
}
\lstdefinelanguage{Fortran95Abstrakt}{
	language=Fortran95Konkret,
	moredelim=[s][\color{purple}]{<}{>},  % Platzhalter für Variablentypen etc. in spitzen Klammern faerben
}
\lstset{
  language=Fortran95Abstrakt,
  basicstyle=\ttfamily\footnotesize,
  frame=single,
  xleftmargin=-6pt,
  framexleftmargin=-8pt,
  numbers=left,
  numberstyle=\tiny\color{gray},
  numbersep=-2pt,
  stepnumber=1,
  breaklines=true,
  tabsize=4,
  showstringspaces=false,
  captionpos=t
}
\lstdefinestyle{neutral}{
  % lstset wird übernommen
  language=bash,
  commentstyle=\color{black},
  stringstyle=\color{black},
  keywordstyle=\color{black}, 
  morekeywords={}
}

% Listing soll in Caption als 'Code' bezeichnet werden
\renewcommand{\lstlistingname}{Code}

% INLINE: \lstinline|<code>|
% UMGEBUNG: \begin{lstlisting}[caption={[<shortcaption>]<Caption>}, label=<label>] <codeblock> \end{lstlisting}

% für 1 Link
\usepackage{hyperref}
\hypersetup{
	colorlinks=true,
	linkcolor=purple,
	urlcolor=teal,
	pdftitle={Fortran Cheat Sheet}
}
\usepackage{cleveref}
\crefname{section}{Abschnitt}{Abschnitte}
\crefname{listing}{Code}{Codes}

\begin{document}

% Überschrift
{\Huge \textbf{\textsf{Fortran Cheat Sheet}}}
Zusammenstellung: Karl Hänsel, Stand: \today

\section{Formalia, Variablen und Konstanten}
\textbf{Allgemein im Programmtext:} nur ASCII-Zeichen (keine Umlaute oder ß), Groß- oder Kleinschreibung Fortran egal, eine Zeile (höchstens 72 Zeichen) pro Befehl, Zeilenfortsetzung mit \lstinline|&|\\
\textbf{Namen allgemein:} beginnen mit Buchstaben, es folgen Buchstaben, Ziffern oder Unterstriche, immer \lstinline|implicit none| vor Deklaration setzen und so implizite Typvereinbarung vermeiden, Konstanten komplett in Groß-, Variablen in Kleinbuchstaben\\
\textbf{Namensbeginn je nach Typ:} logicals \lstinline|l|, character \lstinline|y|, loop indices \lstinline|j|, local integers \lstinline|i|, subprogram integer arguments \lstinline|k|, local reals \lstinline|z|, subprogram real arguments \lstinline|p|, complex numbers \lstinline|c|\\
\textbf{Deklaration von Variablen:} \lstinline|<type>| \lstinline|:: <varname>|\\
\textbf{Zuweisung von Variablen:} \lstinline|<varname>| \lstinline|= <Wert>|\\
\textbf{Konstanten:} nutzen, sobald ein Wert mehrfach auftaucht, um später leichter Programmanpassungen oder -erweiterungen an nur einem Ort vornehmen zu können, Benennung immer komplett in Großbuchstaben\\
\textbf{Definition von Konstanten direkt im Deklarationsteil möglich:} \lstinline|<type>, parameter :: <CONSTNAME>=<Wert>|

\section{Operatoren}
\textbf{Hierarchie der Operatoren:}
\begin{enumerate}
  \item Geklammerte Ausdrücke \lstinline|( )|
  \item Punktrechnung \lstinline|*, /|
  \item Strichrechnung \lstinline|+, -|
  \item Vergleichsoperatoren \lstinline[style=neutral]|==, /=, >, >=, <, <=|
  \item Negation \lstinline|.not.|
  \item Konjunktion \lstinline|.and.|
  \item Disjunktion und Kontravalenz \lstinline|.or., .xor.|
  \item Äquivalenz und Antivalenz \lstinline|.eqv., .neqv.|
\end{enumerate}
\textbf{Verkettung:} mit \lstinline|//|\\
\textbf{Assoziativität der Operatoren:} alle linksassoziativ, nur Potenz-Operator \lstinline|**| rechtsassoziativ, \textbf{CAVE:} wegen Rundungsfehlern nicht immer exakt gleich zur mathematischen Assoziativität\\
\textbf{Potenz} mit \lstinline|**| ungenauer als reine Multiplikation\\
Werden \textbf{Zahlen verschiedener Art} verrechnet, wird die höhere Priorität übernommen, Hierarchie \lstinline|complex > real > integer|\\
\textbf{Vergleichsoperatoren} funktionieren für alle Zahlen und auch character-Variablen (Vergleich des ASCII-Codes), bei komplexen Zahlen nur \lstinline|==, /=|

\section{Zahlen (integer, real, complex)}
\textbf{Ganze Zahlen:} \lstinline|integer| standardmäßig \lstinline|integer(kind=4)| mit Tiefe von 32 Bit, also Wertebereich von $-2^{31}$ bis $2^{31}-1$, bzw. von $-2147483648$ bis $2147483647$, 64 Bit mit \lstinline|integer(kind=8)| (möglich sind 1, 2, 4 und 8; für Details \href{http://earth.uni-muenster.de/~joergs/doc/f90/unix-um/dfum_034.html}{siehe hier})\\
\textbf{Rest der Division/Modulo:} \lstinline|mod(<number 1>,<number 2>)| (Vorzeichen entspricht dem von \lstinline|<number 1>|) \\
\textbf{Umwandeln \lstinline|integer| auf \lstinline|real|:} \lstinline|<real> = real(<integer>)|\\
\textbf{Runden von \lstinline|real| auf nächsten \lstinline|integer|:} \lstinline|<integer> = nint(<real>)| (im Gegensatz zu \lstinline|<integer> = <real>|, wo nur rationaler Teil abgeschnitten wird, \textbf{CAVE!})\\
\textbf{Faustregel zur Wahl des Zahlentyps:} \lstinline|integer| nur, wenn es um Anzahlen oder Indizes geht; soll gerechnet werden, dann als \lstinline|real|\\
\textbf{Komplexe Zahlen:} nach \lstinline|complex :: c<varname>| Zuweisung z.B. mit \lstinline|c<varname>=(1.0,1.0)| (hier $1+\textbf{i}$)\\
\textbf{Absolutbetrag einer Zahl:} \lstinline|abs(<number>)|
\begin{lstlisting}[caption={\bfseries Beispiel: CAVE - implizite Umwandlungen bei Division}]
  real    :: z
  integer :: i
  i = 57
  i = i/10  ! 5
  i = i/10.  ! 5, obwohl i/10.==5.7
  z = i/10  ! 5.0, da i/10 == 5
  z = i/10.  ! 5.7
\end{lstlisting}

\section{Zeichenketten (Strings)}
\textbf{Strings:} in Fortran mit festzulegender Länge an Zeichen  (höchstens 256) \lstinline|character(len=<integer>) :: <character>|\\
\textbf{String-Verkettung:} \lstinline|y_string = "String 1"   // "String 2"|
\begin{lstlisting}[caption={\bfseries Beispiel: String-Verkettung über Code-Zeilen hinweg},label=lst:stringkette]
  y_string = "String 1" // &
             "String 2"
\end{lstlisting}
\textbf{Strings ändern/Teilabfrage:} Strings können auch indiziert werden, z.B. \lstinline|<character>(2:4)  = "neu"|\\
\textbf{Strings aus Wiederholung erstellen:} \lstinline|<character> = repeat(<character>, <integercount>)|, \lstinline|<integercount>| $\geq 0$\\
\textbf{Leerzeichen String-Ende entfernen:} \lstinline|trim(<character>)| \\
\textbf{Strings aus/mit numerischen Variablen generieren:} über write-Befehl mit Format (siehe \cref{sec:formatierung})

\section{Datenverbund (Derived Type)}\label{sec:datenverbund}
\textbf{Ziel:} eigene strukturierte Datentypen definieren und nutzen, die aus mehreren Untervariablen zusammengesetzt sind (auch vorher bereits definierte andere eigene Datenverbünde möglich)\\
\textbf{Ansteuern einzelner Untervariablen} mittels \lstinline|%|, z.B. \lstinline|person%y_name = "Max Musterfrau"| oder Argumentenliste in Deklarationsreihenfolge \lstinline|<typevarname> = (<var1>, <var2>, <var3>...)|\\
\textbf{Einlesen:} sind pro Zeile (oder Eingabe) genau die angegebenen Untervariablen enthalten, reicht zum einlesen \lstinline|read(<unit number>, *)| \lstinline|<typevarname>|\\
\textbf{Überladen von Operatoren:} Operatoren (\lstinline|-, +, ...|) können mit einer Funktion und einem zugehörigen Interface für den Datentyp neu definiert (\glqq überladen\grqq) werden:
\begin{lstlisting}[caption={Beispiel Nutzung eigener Datentypen inklusive Überladung}]
  module punkt2D
    type :: p2
	  real :: z_x
	  real :: z_y
    end type
	interface operator (-)
	  procedure diff
	end interface operator (-)
	contains
	function diff(pa, pb)
	  type(p2), intent(in) :: pa, pb
	  type(p2)             :: diff
	  diff%z_x = pa%z_x - pb%z_x
	  diff%z_y = pa%z_y - pb%z_y
	end function diff
  end module
	
  program punkttest
	use punkt2D
	implicit none
	type(p2) :: punkt_a, punkt_b, punkt_diff
	punkt_a = p2( 2, 3)
	punkt_b = p2(20,30)
	punkt_diff = punkt_a - punkt_b
  end program punkttest
\end{lstlisting}


\section{Verzweigungen/Kontrollstrukturen}
\begin{lstlisting}[caption={\bfseries Verzweigungen mit if}]
  ! einzeilige if-Anweisung
  if (<logical>) <einzelne Anweisung>
 
  ! if-then
  if (<logical>) then
    <Anweisung(en)>
  endif
  
  ! if-then-else
  if (<logical>) then
    <Anweisung(en) 1>
  else
    <Anweisung(en) 2>
  endif
  
  ! if-then-else if-...-else
  if (<logical 1>) then
    <Anweisung(en) 1>
  else if (<logical 2>) then
    <Anweisung(en) 2>
  else if (<logical 3>) then
    <Anweisung(en) 3>
  else
    <Anweisung(en) 4>
  endif
\end{lstlisting}
\begin{lstlisting}[caption={\bfseries Verzweigungen mit select case (um weit verzweigte if-else-Strukturen zu vermeiden)}]
  select case (<integer/logical/character>)
    ! KEIN real zur Ueberpruefung ntuzbar!
    case (<integer/range/logical/character>)
	  <Anweisung(en) 1>
	case (<integer/range/logical/character>)
	  <Anweisung(en) 2>
	case default  ! wenn keiner der obigen
	  <Anweisung(en) 3 sonst>
  end select
  ! auch schachtelbar mit weiteren select case oder if-else-Bloecken
	
  ! Beispiel fuer integer mit Bereichen und mehreren Optionen pro case:
  select case (i_integer)
	case (5:7)    ...
	case (8: )    ...
	case ( :10)    ...
	case ( :0, 10) ...		
	case (12)      ...
  end select
\end{lstlisting}
\textbf{CAVE:} Fälle werden von oben nach unten geprüft, also muss speziellste Prüfung oben stehen, sonst wird darunter ggf. nicht mehr geprüft

\section{Schleifen}
\begin{lstlisting}[caption={\bfseries do-Schleifen}]
  ! do-Zaehlschleife (Schrittweite Standard 1, kann auch negativ sein)
  do <loop integer> = <start integer>, <end integer> [, <step integer>]
    <Anweisung(en)>
  enddo
  
  ! do-while-Schleife
  do while (<logical>)
    <Anweisung(en)>
    ! Wdh. bis <logical> .eqv. .false.
  enddo
  
  ! do-Schleife (CAVE!)
  do
    <Anweisung(en)>
    ! Anweisungen immer wdh. bis...
    if (<logical>) exit ! forciert Ende
  enddo
  
  ! implizite do-Schleife, Beispiel:
  write(*,*) i_array(j), j=1, I_MAX
\end{lstlisting}
\textbf{Schleifensteuerung:} \lstinline|exit| sorgt für sofortiges Beenden der Schleife (sind Schleifen verschachtelt und nicht benannt, wird nur aktueller Schleifenkontext verlassen), \lstinline|cycle| sorgt für sofortigen Sprung zum Anfang der Schleife 
\begin{lstlisting}[caption={\bfseries Beispiel: Benannte do-Schleifen und explizites exit}]
  outer: do j_outer = 0, 5
    middle: do j_middle = 1, 4
      if (<logical>) exit outer
      inner: do j_inner = 3, 2, -1
        <Anweisung(en)>
      enddo inner
    enddo middle
  enddo outer
\end{lstlisting}
\textbf{Benennung von Schleifen nicht notwendig}, wenn kein explizites \lstinline|exit| benötigt wird, aber fördert Lesbarkeit\\
\textbf{Faustregel für Schleifenauswahl:} (konkret oder abstrakt aus vorherigem Verlauf) bekannt wie oft genau etwas gemacht werden muss: do-Zählschleife, nicht (genau) bekannt: do-while-Schleife.

\section{Felder (Arrays)}
\textbf{Variablentyp:} für alle Werte in einem Feld  gleich\\
\textbf{Größe:} unveränderlich, bestehend aus \textbf{rank} (Anzahl Dimensionen), \textbf{shape}s (Anzahl Elemente pro Dimension) und \textbf{extent}s (Integer-Bereich, den die Indizes abdecken pro Dimension), dabei \textbf{höchstens 7 Dimensionen} (nicht unbedingt gleicher shape)\\
\textbf{Nur shape \lstinline|n| angeben:} extent automatisch \lstinline|1:n|\\
\textbf{Nur extent \lstinline|m:n| angegeben:} shape automatisch $n-m+1$\\
\textbf{Deklaration in Unterprogrammen:} hier kann letzte Dimension ggf. ohne \lstinline|shape|-Angabe bleiben (z.B. \lstinline|<type>, dimension(:) :: <varname>|, siehe \cref{sec:unterprogramme})
\begin{lstlisting}[caption={\bfseries Deklarationsmöglichkeiten für arrays}]
  ! EINE DIMENSION (Vektor)
  ! z.B. nur shape=n setzen, Start-Index 1:
  <type>, dimension(n) :: <varname>
  
  ! ZWEI DIMENSIONEN (2D-Matrix)
  ! z.B. shapes mit extents setzen:
  <type>, dimension(m1:n1, m2:n2) ::        <varname>
  
  ! ALTERNATIV auch Deklaration rechts
  <type> :: <varname>(m1:n1, m2:n2)
\end{lstlisting}
\textbf{Arrays zuweisen (eindimensional):} \lstinline|<array>(:) = [<value1>, <value2> ...]| (alt \lstinline|(/.../)|), Zuweisung über implizites \lstinline|do| mit \lstinline|<array>(:) = [(j, j=<integer>, <integer>)]|\\
\textbf{Arrays zuweisen (mehrdimensional):} Standardwert für alle Felder \lstinline|<array>(:,:,:,...) = <value>|, Nutzung der Standard-Funktion \lstinline|reshape| unter Angabe der Form, z.B. für 2 x 2 Matrix \lstinline|<array>(2,2) = reshape([<value11>, <value12>, <value21>, <value22>], [2,2])| (zeilenweise)\\
\textbf{Zuweisung direkt mit Deklaration möglich:} z.B. \lstinline|character(len=2), dimension(7) :: y_wochentage=["Mo", "Di", "Mi", "Do", "Fr", "Sa", "So"]|\\
\textbf{Indizierung von arrays:} \lstinline|<arrayname>(x)| für Element \lstinline|x|, \lstinline|<arrayname>(r:s)| für Elemente \lstinline|r| bis \lstinline|s|, \lstinline|<arrayname>(:, x)| für die Elemente \lstinline|x| aus der gesamten ersten Dimension etc.\\
\textbf{Initialisierung von arrays:} kann im Deklarationsteil oder später erfolgen \lstinline|<arrayname>(:) = 0| setzt alle Elemente auf 0 (\lstinline|<arrayname>| \lstinline|= 0| geht auch, schadet aber der Unterscheidbarkeit von dimensionslosen Variablen und arrays)\\
\textbf{Berechnungen} zwischen arrays und Skalaren sowie mehreren arrays (wenn rank und shpaes übereinstimmen) sind möglich\\
\textbf{CAVE:} \textbf{Index der ersten Dimension wird am schnellsten durchlaufen} und sollte daher in geschachtelten do-Zählschleifen \textbf{immer in der innersten Schleife} durchlaufen werden\\
\textbf{Standard-Funktionen für Felder:} \lstinline|dot_product(<array1>, <array2>)| für Skalarprodukt, \lstinline|maxval(<array>), minval(<array>)| geben Minimum und Maximum für Zahlen zurück, \lstinline|maxloc(<array>), minloc(<array>)| die entsprechenden Indizes als Felder, \lstinline|sum(<array>), size(<array>)| geben Summe und Anzahl aller Feldwerte zurück
\begin{lstlisting}[caption={\bfseries Allocatable Arrays (shape erst im Programmverlauf)}]
  implicit none
  ! Anzahl Dimensionen muss bekannt sein
  real, allocatable, dimension(:,:) :: za_array
  integer :: i_alloc_stat ! optional
  
  ! genaue Deklaration spaeter im Programm, optional mit Fehlervariable zur Abfrage hinterher
  allocate(za_array(<integer1>, <integer2>), stat=i_alloc_stat)
  if (i_alloc_stat /= 0) write(*,*) "Fehler, Speicherplatz reicht nicht aus!"
  ! Abfrage, ob bereits Speicherplatz zugewiesen wurde
  if (allocated(za_array)) write(*,*) "erfolgreich!"
  ! optional: Speicherplatz freigeben
  deallocate(za_array)
\end{lstlisting}
\begin{lstlisting}[caption={\bfseries where-Statement als Ersatz für do-if in Feldern},language=Fortran95Konkret]
	! statt ...
	do j=1, i_dim
		if (z_array1(i) <= 0.0) z_array2(i) = 0.0
	enddo
	! geht es eleganter in einer Zeile:
	where (z_array1(:) < 0.0) z_array2(:) = 0.0
	! auch innerhalb des gleichen arrays:
	where (z_array(:) < 0.0) z_array(:) = 0.0
	! ...oder verschachtelt mit elsewhere:
	where (i_array(:) < 42)
	  i_array(:) = 0
	elsewhere
	  i_array(:) = 42
	endwhere
	! ...oder mit logischer Maske:
	where (l_logische_maske(:))
	  y_array(:) = "hier wahr"
	elsewhere
	  y_array(:) = "hier falsch"
	endwhere
	! es koennen auch mehrere where-Anweisungen verschachtelt werden
\end{lstlisting}
\textbf{CAVE:} Das gleichzeitige Abarbeiten zweier Arrays oder die Prüfung auf eine logische Maske sind nur möglich, wenn alle beteiligten Arrays gleichen \lstinline|rank| und \lstinline|shape| haben!
\begin{lstlisting}[caption={forall-Statement als Parallelisierung von where}]
	! forall mit Wertebereich (ggf. Schrittweite) einer Laufvariable und ggf. eine logische Maske, die anzuwenden ist, z.B.:
	forall(j=<integer1>:<integer2>, <logical expression>) za_array(j) = <real>
	
	! es koennen auch mehrere Bereiche genutz werden, z.B. fuer Transposition einer Matrix:
	forall(ja=1:I_N, jb=1:I_N)
	  i_matrix(ja, jb) ) = i_matrix(jb, ja)
	end forall
	! funktioniert korrekt, da Augabe parallelisiert wird, d.h. es benoetigt keinen Zwischenspeicher fuer Aenderungen
\end{lstlisting}

\section{Ein- und Ausgabe}
\subsection{Standardeingabe und -ausgabe}
\textbf{Einlesen von Variablen über Terminal:} \lstinline|read(*,*) <varname>|\\
\textbf{CAVE:} beim Einlesen von input sorgt Eingabe von \lstinline|/| für ein Ignorieren aller weiteren Eingaben in der Zeile, ggf. muss also \lstinline|'/'| genutzt werden\\
\textbf{Ausgabe von Variablen über Terminal:} \lstinline|write(*,*) <Wert>|\\
\textbf{Formatierung beeinflussen:} \lstinline|(*,<character>)|, Syntax siehe \cref{sec:formatierung}\\
\textbf{Umleitung von input und output beim Programmaufruf:} alles, was über die Standard-Ausgabe (erster \lstinline|*|) eingelesenen oder geschrieben wird, kann auch aus einer Datei eingelesen oder in eine Datei geschrieben werden, indem beim Aufruf in das Programm umgeleitet wird (\lstinline[style=neutral]|program.x < inputfile|), bzw. aus dem Programm umgeleitet wird (\lstinline[style=neutral]|program.x > outputfile|) werden (oder beides kombiniert: \lstinline[style=neutral]|program.x < inputfile > outputfile|)

\subsection{Dateien öffnen/schließen}\label{sec:dateien-oeffnen}
\textbf{Existenzprüfung:} gibt an, ob eine Datei existiert \lstinline|inquire(file=<character>, exist=<logicalname>)| \\
\textbf{Datei öffnen (Pflichtargumente):} \lstinline|open(unit=<integer>,file=<character>)|\\
\textbf{\lstinline|unit|:} nicht-negativer, zweistelliger \lstinline|integer| (einstellig könnten Peripheriegeräte sein)\\
\textbf{\lstinline|file|:} Dateiname wenn im gleichen Ordner, sonst mit Pfad\\
\textbf{Status:} \lstinline|status=<status>| mit Möglichkeiten \lstinline|"old", "new", "unknwon" (default), "replace", "scratch"| (scratch: nur temporär während Programmlauf existent)\\
\textbf{Aktion:} \lstinline|action=<action>| mit den Möglichkeiten \lstinline|"read", "write", "readwrite" (default)|\\
\textbf{Fehlervariable:} \lstinline|iostat=<integername>|, ist \lstinline|<integername>| $\neq 0$, gab es Fehler, z.B. EOF (End of File), \textbf{immer prüfen!}\\
\textbf{Position:} \lstinline|position="append"|, wenn am Ende angefügt statt von Anfang überschrieben \lstinline|(default)| werden soll\\
\textbf{Format:} mit Format (ASCII; Standard) oder ohne Format (Binärdatei, Angabe mit \lstinline|form="unformatted"|)\\
\textbf{Direktzugriff:} spezielle Datensätze wie Datenbanken, bei denen alle Records gleich lang sind (record byte length \lstinline|recl| für alle gleich), können über Direktzugriff angesprochen werden (kein sukzessives Einlesen/Schreiben nötig, sondern direkter Zugriff auf Record-Nummer mit \lstinline|write(<unit>, rec=<integer>), bzw. read(<unit>, rec=<integer>)|) mit den Argumenten \lstinline|access="direct", recl=<integer>, form="unformatted"|\\
\textbf{Datei schließen:} \lstinline|close(unit=<integer>)|, eigentlich automatisch am Programmende, insbesondere bei großen Datenmengen aber sinnvoll, sobald nicht mehr benötigt

\subsection{Dateien lesen/schreiben}
\textbf{Formatierte Datei lesen/schreiben:} \lstinline|write(unit=<integer>, *)|, bzw. \lstinline|read(unit=<integer>, *)| oder alternativ mit expliziter Format-Angabe statt \lstinline|*| (siehe \cref{sec:formatierung}), aber
\textbf{Einlesen von Dateien sollte in Standard-Formatierung \lstinline|*| erfolgen} (\lstinline|read(unit=<integer>, *)|), da Formatierung fehleranfällig beim Einlesen\\
\textbf{Unformatierte Datei lesen/schreiben:} jede Anweisung entspricht einem Record von variabler Byte-Größe (z.B. können auch Felder übergeben werden), Anweisungen ohne Formatangabe, nur Datei-Unit (\lstinline|write(unit=<integer>) <var>|, bzw. \lstinline|read(uni=<integer>) <var>|)\\
\textbf{Zurücksetzen:} \lstinline|rewind(unit=<integer>)| setzt die Position der Datei wieder auf Anfang zum erneuten Lesen/Schreiben\\
\textbf{Einen Record/Zeile zurück:} \lstinline|backspace(unit=<integer>)| setzt Position der Datei um einen record, bzw. Zeile zurück

\subsection{Formatierung}\label{sec:formatierung}
\textbf{Grundform Format-String:} \lstinline|"(<character>)"|\\
\textbf{Mögliche Zahlenangaben:}  Datenfeldweite (muss groß genug sein, Dezimalpunkt zählt mit, unten \lstinline|w|), Anzahl Dezimalstellen (unten \lstinline|d|), Anzahl Ziffern im Exponenten (unten \lstinline|e|), Anzahl ausgegebener Ziffern (ggf. führende Nullen hinzugefügt, unten \lstinline|m|)\\
\textbf{\lstinline|integer|:} \lstinline|Iw[.m]|\\
\textbf{\lstinline|real|:} Fixpunktdarstellung: \lstinline|Fw.d| (wobei \lstinline|w|$\geq$\lstinline|d|$+2$), Exponentialdarstellung: \lstinline|Ew.d| (wobei \lstinline|w|$\geq$\lstinline|d|$+7$), Exponentialdarstellung mit fester Exponentenlänge: \lstinline|Ew.dEe| (wobei \lstinline|w|$\geq$\lstinline|d|$+$\lstinline|e|$+5$), Exponentialdarstellung mit fester Stellenzahl \lstinline|n| vor dem Komma: \lstinline|nPEw.d|, automatische Wahl von \lstinline|F| oder \lstinline|E| Format: \lstinline|Gw.d|\\
\textbf{\lstinline|character|:} \lstinline|A[w]|\\
\textbf{\lstinline|logical|:} \lstinline|Lw|\\
\textbf{Positionierung in \lstinline|n|ter Spalte (Tabulator):} \lstinline|Tn|\\
\textbf{Ausgabe von \lstinline|n| Leerstellen:} \lstinline|nX|\\
\textbf{Zeilensprung:} \lstinline|/|\\
\textbf{Zeilenvorschub unterdrücken:} \lstinline|"(<character>$)"|\\
\textbf{\lstinline|n|-fach wiederholen:} \lstinline|n(<character>)|\\
\textbf{Auslagerung in Variable:} sinnvoll, wenn mehrfach verwendet oder  zwischendurch verändert werden soll\\
\textbf{Schreiben auf, bzw. Umwandeln von Variablen:} insbesondere kann Text in Zahlen (und umgekehrt) umgewandelt werden, indem \lstinline|write| und \lstinline|read| auf Variablen angewendet werden, z.B. \lstinline|read(y_datum, "(2I, '.', 2I, '.', 4I)" i_tag, i_monat, i_jahr|\\
\textbf{Standard-Funktionen für Text-Formatierung:} \lstinline|trim(<character>)| entfernt Leerzeichen am Ende, \lstinline|adjustl(<character>), adjustr(<character>)| richten Text links-, bzw. rechtsbündig aus, \lstinline|len(<character>), len_trim(<character>)| geben Länge zurück (ggf. ohne Leerzeichen am Ende), \lstinline|index(<character1>, <character2>)| gibt Position zurück, an der Muster \lstinline|<character2>| in \lstinline|<character1>| auftaucht

\subsection{Parameterübergabe bei Programmaufruf}
\textbf{Anzahl der übergebenen Kommandozeilenparameter abfragen:} \lstinline|<integer> = command_argument_count()|\\
\textbf{Parameter einlesen:} \lstinline|call get_command_argument(<integer>, <charactervariable>)| (dabei \lstinline|<integer>|$\leq$\lstinline|command_argument_count()|), immer als \lstinline|character| eingelesen, Umwandlung z.B. über \lstinline|read(<charactervariable>,*) <numericalvariable>|\\
\textbf{Programmaufruf:} z.B. für zwei \lstinline|integer|-Argumente \lstinline[language=bash]|<programname>.x <integer1> <integer2>|

\subsection{Variablenübergabe durch Namelists}
\textbf{Variablen-Deklaration im Programm:} erfolgt wie gewohnt, durch namelist einzulesende Werte werden bereits im Deklarationsteil mit Standard-Wert belegt\\
\textbf{Namelist:} enthält nur Angaben zu den Variablen, die von Standard-Belegung abweichen sollen (kann also auch leer sein und es können für unterschiedliche Fälle unterschiedliche Namelists vorliegen)\\
\textbf{Namelist-Deklaration:} unter den Deklarationen der zugehörigen Variablen mit \lstinline|namelist /<namelistname>/ <var1>, <var2>, ...|\\
\textbf{Öffnen und Lesen der Namelist:} Öffnen wie alle Dateien (siehe \cref{sec:dateien-oeffnen}), Einlesen der enthaltenen Variablen durch \lstinline|read(<unitinteger>,nml=<namelistname>)|
\begin{lstlisting}[caption={Struktur Namelist-Datei <namelistname>.nml}]
	&<namelistname>
	<var1> = <value1>
	<var2> = <value2>
	...
	/
\end{lstlisting}

\section{Unterprogramme und Module}\label{sec:unterprogramme}
\textbf{Speichern:} entweder alles in einer Datei mit Hauptprogramm oder einzelne Dateien möglich\\
\textbf{Subroutinen:} \lstinline|subroutine| führt Code aus, nutzt dazu die als Argumente übergebenen Werte, Aufruf mit \lstinline|call <subroutine>(<Argument(e)>)|\\
\textbf{Funktionen:} \lstinline|<type> function| gibt einen Wert der Art \lstinline|<type>| zurück, nutzt dazu die als Argumente übergebenen Werte, Zuweisung mit \lstinline|<varname>| \lstinline|= <function>(<Argument(e)>)|, muss im Hauptprogramm vorher mit entsprechendem Datentyp definiert werden: \lstinline|<type>| \lstinline|:: <function>|\\
\textbf{Rekursive Funktionen:} entweder über Definitionskopf \lstinline|recursive function <funcname>(<Arugement(e)>) result(<resultvarname>)| und dann mit \lstinline|<resultvartype> :: <resultvarname>| im Deklarationsteil oder mit \lstinline|<resulttype> recursive function <funcname>(<Arugment(e)>) result(<resultvarname>)| ohne explizite Deklaration später; innerhalb der Funktionsdefinition wird dann irgendwo die Funktion selbst wieder aufgerufen (an Abbruchkriterien denken!)\\
\textbf{Deklarationen in Unterprogrammen:}  angeben, ob ein Argument nur eingelesen und nicht verändert (\lstinline|intent(in)|), unabhängig vom Originalwert hinterher zurückgegeben (\lstinline|intent(out)|) oder gelesen UND verändert werden soll (\lstinline|intent(inout)|)\\
\textbf{CAVE:} \textbf{Anzahl und Reihenfolge der Argumente} müssen zwischen Deklaration und Aufruf übereinstimmen!\\
\textbf{Variable Längendeklaration:} ist z.B. String-Länge  eines einzugebenden Arguments in Unterprogramm nicht bekannt, kann sie mit \lstinline|character(len=*) :: <Argument>| deklariert werden\\
\textbf{Felddeklaration in Unterprogrammen:} letzte Dimension kann ohne explizite Größenangabe bleiben, also z.B. \lstinline|real, dimension(<integer>, *) :: za_zweidim| oder auch \lstinline|real, dimension(*) :: za_eindim|\\
\textbf{CAVE: sinnvolle Fehlerausgaben bei falscher Datentyp-Übergabe hängen vom Compiler ab}, soll eine Warnung sichergestellt werden, müssen Unterprogramme in Modulen oder über ein Interface (s.u.) definiert werden\\
\textbf{Modul:} enthält Programminhalt (Deklarationen, Wertzuweisungen, aber auch subroutines, functions oder interfaces), der von allen Programmteilen oder mehreren Programmen genutzt wird\\
\textbf{Einbindung Modul in (Unter-)Programme:} \lstinline|use <modulename>|, bei Einzelkompilierung Modul zuerst kompilieren\\
\textbf{CAVE:} alle Unterprogramme haben auf den \textbf{aktuellen Zustand einer Modul-Variable} im Programmverlauf Zugriff (auch wenn gemeinsames Modul eigentlich nur initialisiert)\\
\textbf{Lokale Variablen:} wird eine Variable nicht über ein Modul \glqq verbreitet\grqq, gilt sie immer nur lokal in der jeweiligen Programmeinheit (Programm, Subroutine, Funktion)\\
\textbf{Interfaces:} \lstinline|interface| kann helfen, Fehlermeldung bei falscher Datentypübergabe an Unterprogramme zu forcieren, in Unterprogrammen optionale Argumente und Allokation zu ermöglichen oder Operatoren bei selbst definierten Datentypen zu überladen (siehe \cref{sec:datenverbund})\\
\textbf{Deklaration von Interfaces:} kann im Hauptprogramm stehen oder in (alleinstehendes!) Modul (getrennt von Deklaration von Konstanten oder Initialisierung von Variablen für Haupt- und Unterprogramme) ausgelagert werden, enthalten nur Informationen, die für Unterprogrammnutzung erforderlich sind (Unterprogrammname, Parameterdeklarationen)\\
\textbf{Optionale Parameter und Allokieren für Unterprogramme durch Interfaces:} über \lstinline|optional|, bzw. \lstinline|allocatable| in der Deklaration (siehe \cref{code:structure}), es gilt beim Aufruf dennoch Reihenfolge der Argumente, demzufolge müssen ggf. Argumente explizit angegeben werden (also \lstinline|<subname>(<argname>=<argvalue>)|)\\
\begin{lstlisting}[caption={\bfseries Dateischema Modul, Haupt- und Unterprogramme },label=code:structure]
  module <modulename1>
  ! erklaerender Kommentar
  implicit none
  <globale Deklarationen, Definitionen, Initialisierungen>
  
  contains
  <globale Prozeduren, Funktionen, Interfaces, falls nicht anderswo definiert>
  end module <modulename>
  
  module <modulename2>
  ! erklaerender Kommentar
  interface
    subroutine <suname>(<Argument(e))
      <type>, intent(<intent>) :: <argname>
      <type>, allocatable, dimension(:,:,...) :: <allocargname>
      <type>, intent(<intent>), optional :: <optargname>
      ...
    end subroutine <subname>
  end interface
  end module <modulename>
  
  program <progname>
  ! erklaerender Kommentar
  use <modulename1>
  use <modulename2>
  implicit none
  <lokale Deklarationen (auch zu nutzende Funktionen), Definitionen, Initialisierungen>
  <Anweisungsteil>
  stop
  end program <progname>
  
  <type> function <funcname>(<Arugment(e)>)
  ! erklaerender Kommentar
  use <modulename1>
  implicit none
  <type>, intent(<intent>) :: <argname>
  <weitere lokale Deklarationen, Definitionen, Initialisierungen>
  <Anweisungen>
  return
  end function <funcname>
  
  subroutine <subname>(<Arguemt(e)>)
  ! erklaerender Kommentar
  use <modulename1>
  implicit none
  <type>, intent(<intent>) :: <argname>
  <type>, allocatable, dimension(:,:,...) :: <allocargname>
  <type>, intent(<intent>), optional :: <optargname>
  <weitere lokale Deklarationen, Definitionen, Initialisierungen>
  <Anweisungen>
  return
  end subroutine <subname>
\end{lstlisting}

\section{Kompilieren}
\textbf{Einfach:} \lstinline[style=neutral]|gfortran -o <progname>.x <progname>.f90| \\
\textbf{Mehr Fehlermeldungen vorab} (z.B. bei Division durch 0 oder Verwendung von Variablen vor Wertzuweisung): \lstinline[style=neutral]|gfortran -ffpe-trap=invalid,zero,overflow -finit-real=snan <progname>.x <progname>.f90|\\
\textbf{Bei Unterprogrammnutzung:} \lstinline[style=neutral]|gfortran -o <progname>.x <progname>.f90 <subprogname>.f90| oder alternativ:
\begin{lstlisting}[style=neutral,caption={\textbf{Kompilieren von Haupt- und Unterprogrammenn}}]
  gfortran -c <modulename>.f90  ! zuerst!
  gfortran -c <progname>.f90 
  gfortran -c <subprogname>.f90 
  gfortran -o <progname>.x <programe>.o <subprogname>.o <modulename>.o
\end{lstlisting}
\textbf{Temporäre Dateien:} die bei Einzelkompilierung von Unterprogrammen und Modulen erzeugten Dateien werden nach Kompilierung nicht mehr benötigt und können gelöscht werden
\end{document}
